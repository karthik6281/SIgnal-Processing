\documentclass[journal,12pt,twocolumn]{IEEEtran}
\usepackage{setspace}
\usepackage{gensymb}
\usepackage{xcolor}
\usepackage{caption}
\singlespacing
\usepackage{siunitx}
\usepackage[cmex10]{amsmath}
\usepackage{mathtools}
\usepackage{hyperref}
\usepackage{amsthm}
\usepackage{mathrsfs}
\usepackage{txfonts}
\usepackage{stfloats}
\usepackage{cite}
\usepackage{cases}
\usepackage{subfig}
\usepackage{longtable}
\usepackage{multirow}
\usepackage{enumitem}
\usepackage{mathtools}
\usepackage{listings}
\usepackage{tikz}
\usetikzlibrary{shapes,arrows,positioning}
\usepackage{circuitikz}
\let\vec\mathbf
\DeclareMathOperator*{\Res}{Res}
\renewcommand\thesection{\arabic{section}}
\renewcommand\thesubsection{\thesection.\arabic{subsection}}
\renewcommand\thesubsubsection{\thesubsection.\arabic{subsubsection}}

\renewcommand\thesectiondis{\arabic{section}}
\renewcommand\thesubsectiondis{\thesectiondis.\arabic{subsection}}
\renewcommand\thesubsubsectiondis{\thesubsectiondis.\arabic{subsubsection}}
\hyphenation{op-tical net-works semi-conduc-tor}

\lstset{
	language=Python,
	frame=single, 
	breaklines=true,
	columns=fullflexible
}
\begin{document}
	\theoremstyle{definition}
	\newtheorem{theorem}{Theorem}[section]
	\newtheorem{problem}{Problem}
	\newtheorem{proposition}{Proposition}[section]
	\newtheorem{lemma}{Lemma}[section]
	\newtheorem{corollary}[theorem]{Corollary}
	\newtheorem{example}{Example}[section]
	\newtheorem{definition}{Definition}[section]
	\newcommand{\BEQA}{\begin{eqnarray}}
		\newcommand{\EEQA}{\end{eqnarray}}
	\newcommand{\define}{\stackrel{\triangle}{=}}
	\newcommand{\myvec}[1]{\ensuremath{\begin{pmatrix}#1\end{pmatrix}}}
	\newcommand{\mydet}[1]{\ensuremath{\begin{vmatrix}#1\end{vmatrix}}}
	\bibliographystyle{IEEEtran}
	\providecommand{\nCr}[2]{\,^{#1}C_{#2}} % nCr
	\providecommand{\nPr}[2]{\,^{#1}P_{#2}} % nPr
	\providecommand{\mbf}{\mathbf}
	\providecommand{\pr}[1]{\ensuremath{\Pr\left(#1\right)}}
	\providecommand{\qfunc}[1]{\ensuremath{Q\left(#1\right)}}
	\providecommand{\sbrak}[1]{\ensuremath{{}\left[#1\right]}}
	\providecommand{\lsbrak}[1]{\ensuremath{{}\left[#1\right.}}
	\providecommand{\rsbrak}[1]{\ensuremath{{}\left.#1\right]}}
	\providecommand{\brak}[1]{\ensuremath{\left(#1\right)}}
	\providecommand{\lbrak}[1]{\ensuremath{\left(#1\right.}}
	\providecommand{\rbrak}[1]{\ensuremath{\left.#1\right)}}
	\providecommand{\cbrak}[1]{\ensuremath{\left\{#1\right\}}}
	\providecommand{\lcbrak}[1]{\ensuremath{\left\{#1\right.}}
	\providecommand{\rcbrak}[1]{\ensuremath{\left.#1\right\}}}
	\theoremstyle{remark}
	\newtheorem{rem}{Remark}
	\newcommand{\sgn}{\mathop{\mathrm{sgn}}}
	\newcommand{\rect}{\mathop{\mathrm{rect}}}
	\newcommand{\sinc}{\mathop{\mathrm{sinc}}}
	\providecommand{\abs}[1]{\left\vert#1\right\vert}
	\providecommand{\res}[1]{\Res\displaylimits_{#1}} 
	\providecommand{\norm}[1]{\lVert#1\rVert}
	\providecommand{\mtx}[1]{\mathbf{#1}}
	\providecommand{\mean}[1]{E\left[ #1 \right]}
	\providecommand{\fourier}{\overset{\mathcal{F}}{ \rightleftharpoons}}
	\providecommand{\ztrans}{\overset{\mathcal{Z}}{ \rightleftharpoons}}
	\providecommand{\system}[1]{\overset{\mathcal{#1}}{ \longleftrightarrow}}
	\newcommand{\solution}{\noindent \textbf{Solution: }}
	\providecommand{\dec}[2]{\ensuremath{\overset{#1}{\underset{#2}{\gtrless}}}}
	\numberwithin{equation}{section}
	\makeatletter
	\@addtoreset{figure}{problem}
	\makeatother
	\let\StandardTheFigure\thefigure
	\renewcommand{\thefigure}{\theproblem}
	\def\putbox#1#2#3{\makebox[0in][l]{\makebox[#1][l]{}\raisebox{\baselineskip}[0in][0in]{\raisebox{#2}[0in][0in]{#3}}}}
	\def\rightbox#1{\makebox[0in][r]{#1}}
	\def\centbox#1{\makebox[0in]{#1}}
	\def\topbox#1{\raisebox{-\baselineskip}[0in][0in]{#1}}
	\def\midbox#1{\raisebox{-0.5\baselineskip}[0in][0in]{#1}}
	
	\title{ Digital Signal Processing \\ \Large EE3900 \\ \vspace*{12pt} \textbf{Fourier Series}}
	\author{I Sai Pradeep\\ \normalsize AI21BTECH11013 \\ \vspace*{20pt} \normalsize \today}
	\maketitle 
	\tableofcontents
	\begin{abstract}
		This manual provides a simple introduction to Fourier Series
	\end{abstract}
	\section{Periodic Function}
	Let 
	\begin{align}
		x(t) &= A_0\abs{\sin\brak{2\pi f_0 t}}
		\label{eq:x(t)}
	\end{align}
	\begin{enumerate}[label=\thesection.\arabic*
		,ref=\thesection.\theenumi]
		\item Plot $x(t)$.\\
		\solution 
		\begin{lstlisting}
			wget https://github.com/Pradeep8802/EE3900-Digital-Signal-Processing/blob/main/charger/codes/1.1.py
		\end{lstlisting}
		\begin{figure}[!ht]
			\centering
			\includegraphics[width=\columnwidth]{./figs/1_1.png}
			\caption{}
			%\label{fig:ckt}
		\end{figure}
		\item Show that $x(t)$ is periodic and find its period.
		\solution 
		A signal $x(t)$ is said to be periodic with fundamental period $T$ if
		\begin{align}
			\label{eq:peroidic}
			x(t+nT)=x(t) \forall n \in \mathbb{Z}
		\end{align}
		Let $T$ be fundamental period of $x(t)$. Comparing \eqref{eq:peroidic} and \eqref{eq:x(t)}, we get
		\begin{align}
			A_0\abs{\sin\brak{2\pi f_0 t}}&=A_0\abs{\sin\brak{2\pi f_0 (t+T)}}\\
			\abs{\sin\brak{2\pi f_0 t}}&=\abs{\sin\brak{2\pi f_0 (t+T)}}\\
			\abs{\sin\brak{2\pi f_0 t}}&=\abs{\sin\brak{2\pi f_0 t+2\pi f_0 T)}}
		\end{align}
		As $|sin\theta|$ is periodic with fundamental period $F=\pi$, Hence,
		\begin{align}
			\abs{\sin\brak{t}}=\abs{\sin\brak{t+F)}}
		\end{align}
		Hence,$2\pi f_0  T=\pi$, therefore, fundamental period($T$) is 
		\begin{align}
			\label{eq:ftp}
			T=\frac{\pi}{2 \pi f_0}=\frac{1}{2f_0}
		\end{align}
	\end{enumerate}
	\section{Fourier Series}
	Consider $A_0 =12$ and $f_0 = 50$ for all numerical calculations.
	\begin{enumerate}[label=\thesection.\arabic*,ref=\thesection.\theenumi]
		\item If
		%\cite{proakis_dsp}
		\begin{align}
			x(t) = \sum_{k = -\infty}^{\infty}c_ke^{j2\pi kf_0 t}
			\label{eq:one-Z-complex}
		\end{align}
		show that 
		\begin{align}
			c_k = f_0\int_{-\frac{1}{2f_0}}^{\frac{1}{2f_0}}x(t)e^{-j2\pi kf_0 t}\, dt
			\label{eq:one-Z}
		\end{align}
		\solution 
		From \eqref{eq:one-Z-complex},
		\begin{align}
			\label{eq:sum}
			x(t) = \sum_{k = -\infty}^{\infty}c_ke^{j2\pi kf_0 t}
		\end{align}
		Mulitply $e^{-j2\pi lf_0 t}$ on both sides of \eqref{eq:sum}, we get,
		\begin{align}
			\label{eq:a}
			x(t)e^{-j2\pi lf_0 t}=\sum_{k = -\infty}^{\infty}c_ke^{j2\pi kf_0 t}e^{-j2\pi lf_0 t}
		\end{align}
		Integrating \eqref{eq:a} w.r.t. $t$ from $-T$ to $T$, and $T=\frac{1}{f_0}$, we get,\\
		\begin{align}
			\label{eq:d}
			\int_{-\frac{1}{2f_0}}^{\frac{1}{2f_0}}x(t)e^{-j2\pi kf_0 t}\, dt&=\int_{-\frac{1}{2f_0}}^{\frac{1}{2f_0}}\sum_{k = -\infty}^{\infty}c_ke^{j2\pi \brak{k-l}f_0 t}\,dt\\
			&=\sum_{k = -\infty}^{\infty}c_k\int_{-\frac{1}{2f_0}}^{\frac{1}{2f_0}}e^{j2\pi \brak{k-l}f_0 t}\,dt
		\end{align}
		Consider the following cases.\\
		case-1:$k=l$
		\begin{align}
			\int_{-\frac{1}{2f_0}}^{\frac{1}{2f_0}}e^{j2\pi \brak{k-l}f_0 t}\,dt&=\int_{-\frac{1}{2f_0}}^{\frac{1}{2f_0}}e^{0}\,dt\\
			&=\int_{-\frac{1}{2f_0}}^{\frac{1}{2f_0}} 1 \,dt
		\end{align}
		case-2: $k \neq l$ \\
		Let $n=f_0(k-l)$, here $n \in \mathbb{Z}$
		\begin{align}
			\label{eq:b}
			\int_{-\frac{1}{2f_0}}^{\frac{1}{2f_0}}e^{j2\pi \brak{k-l}f_0 t}\,dt&=\int_{-\frac{1}{2f_0}}^{\frac{1}{2f_0}}e^{2n\pi}\,dt	
		\end{align}
		Here, $2n\pi T=2f_0(k-l)T\pi$, and $2n\pi T=(k-l)\pi$
		\begin{align}
			\int_{-\frac{1}{2f_0}}^{\frac{1}{2f_0}}e^{j2n\pi}\,dt&=\int_{-\frac{1}{2f_0}}^{\frac{1}{2f_0}}1\,dt\\&=\int_{-\frac{1}{2f_0}}^{\frac{1}{2f_0}}\cos(2n\pi)+j\sin(2n\pi) \,dt
		\end{align}
		\begin{align}
			\int_{-\frac{1}{2f_0}}^{\frac{1}{2f_0}}e^{j2n\pi}\,dt&=
			-\sin(2n\pi t)\bigg|_{-\frac{1}{2f_0}}^{\frac{1}{2f_0}}\\&+j\cos(2n\pi t)\bigg|_{-\frac{1}{2f_0}}^{\frac{1}{2f_0}}\\
			&=-\sin(2n\pi t)\bigg|_{-\frac{1}{2f_0}}^{\frac{1}{2f_0}}\\&+j\cos(2n\pi t)\bigg|_{-\frac{1}{2f_0}}^{\frac{1}{2f_0}}\\
			&=-\sin(2n\pi t)\bigg|_{-\frac{1}{2f_0}}^{\frac{1}{2f_0}}\\&+j\cos(2n\pi t)\bigg|_{-\frac{1}{2f_0}}^{\frac{1}{2f_0}}\\
			\label{eq:c}\\
			&=-\sin((k-l)\pi)+\sin(-(k-l)\pi)\\&+j\cos((k-l)\pi)-j\cos(-(k-l)\pi)\\
		\end{align}
		Since $k-l \in \mathbb{Z}$,$\sin((k-l)\pi)=0$ and $\sin(-(k-l)\pi)=0$, simillarly, as $\cos(\theta)=\cos(-\theta)$, we get $\cos((k-l)\pi)-\cos(-(k-l)\pi)=0$\\
		From \eqref{eq:c},
		\begin{align}
			\int_{-\frac{1}{2f_0}}^{\frac{1}{2f_0}}e^{j2n\pi}\,dt&=\int_{-\frac{1}{2f_0}}^{\frac{1}{2f_0}}1\,dt\\&=0+j0=0
		\end{align}
		Hence, we have,
		\begin{align}
			\int_{-\frac{1}{2f_0}}^{\frac{1}{2f_0}}e^{j2\pi \brak{k-l}f_0 t}\,dt=
			\begin{cases}
				0 & k\neq l
				\\
				\int_{-\frac{1}{2f_0}}^{\frac{1}{2f_0}}1\,dt& k=l
			\end{cases}
		\end{align}
		From \eqref{eq:d},
		\begin{align}
			\int_{-\frac{1}{2f_0}}^{\frac{1}{2f_0}}x(t)e^{-j2\pi kf_0 t}\, dt&=\sum_{k = -\infty}^{\infty}c_k\int_{-\frac{1}{2f_0}}^{\frac{1}{2f_0}}e^{j2\pi \brak{k-l}f_0 t}\,dt\\
			&=c_k \times \int_{-\frac{1}{2f_0}}^{\frac{1}{2f_0}}1\,dt
		\end{align}
		\begin{align}
			c_k &= f_0\int_{-\frac{1}{2f_0}}^{\frac{1}{2f_0}}x(t)e^{-j2\pi kf_0 t}\, dt\\
			\therefore c_k &= \frac{2}{T} \int_{-\frac{1}{T}}^{\frac{1}{T}}x(t)e^{-j2\pi kf_0 t}\, dt
		\end{align}
		\item Find $c_k$ for 
		\eqref{eq:x(t)}\\
		\solution
		We know that,
		\begin{align}
			c_k = 2f_0\int_{0}^{\frac{1}{2f_0}}x(t)e^{- j2 \pi kf_0 t}\, dt
		\end{align}
		when $t \in \bigg( 0,\frac{1}{2f_0}\bigg)$, $x(t)=A_0 \sin\brak{2 \pi f_0t}$\\
		\begin{align}
			c_k &= 2f_0\int_{0}^{\frac{1}{2f_0}}A_0 \brak{\frac{e^{j2\pi f_0t}-e^{-j 2\pi f_0t}}{2j}} e^{- j2 \pi kf_0 t}\,dt\\
			&=A_0f_0\int_{0}^{\frac{1}{2f_0}} \brak{\frac{e^{j2\pi\brak{1-k} f_0t}-e^{j 2\pi\brak{-1-k} f_0t}}{j}}\,dt\\
			&=A_0f_0\bigg(\frac{e^{j2\pi\brak{1-k} f_0t}}{-2\pi \brak{1-k}f_0}\bigg|_0^{\frac{1}{2f_0}} \\&- \frac{e^{j2\pi\brak{-1-k} f_0t}}{-2\pi \brak{-1-k}f_0}\bigg|_0^{\frac{1}{2f_0}}\bigg)\\
			&=A_0\sbrak{\frac{e^{j\pi\brak{1-k}}-1}{2\pi\brak{k-1}}-\frac{e^{-j\pi\brak{1+k}}-1}{2\pi\brak{k+1}}}
		\end{align}
		Hence,
		\begin{equation}
			\label{eq:ck}
			c_k= \begin{cases}
				\frac{2A_0}{\pi\brak{1-k^2}}&k=even
				\\
				0&k=odd
			\end{cases}
		\end{equation}
		\item Verify 
		\eqref{eq:x(t)}
		using python.\\
		\solution 
		\begin{lstlisting}
			wget https://github.com/Pradeep8802/EE3900-Digital-Signal-Processing/blob/main/charger/codes/2.3.py
			python3 2.3.py
		\end{lstlisting}
		\begin{figure}[!ht]
			\centering
			\includegraphics[width=\columnwidth]{./figs/2_3.png}
			\caption{}
			%\label{fig:ckt}
		\end{figure}
		\item Show that 
		\begin{align}
			x(t) = \sum_{k = 0}^{\infty}\brak{a_k\cos{2\pi kf_0 t}+b_k\sin{2\pi kf_0 t}}
			\label{eq:one-Z-real}
		\end{align}
		and obtain the formulae for $a_k$ and $b_k$.\\
		\solution
		Using \eqref{eq:one-Z-complex},
		\begin{align}
			x(t) = \sum_{k = -\infty}^{\infty}c_ke^{j2\pi kf_0 t}
		\end{align}
		As,
		\begin{align}
			e^{j2\pi kf_0 t}=\cos\brak{2\pi kf_0 t}+j\sin\brak{2\pi kf_0 t}
		\end{align}
		From \eqref{eq:one-Z-complex}, we have,
		\begin{align}
			x(t) &= \sum_{k = -\infty}^{\infty}c_k\sbrak{\cos\brak{2\pi kf_0 t}+j\sin\brak{2\pi kf_0 t}}\\
			\label{eq:2.4}
			&=\sum_{k = -\infty}^{\infty}c_k\cos\brak{2\pi kf_0 t}+jc_k\sin\brak{2\pi kf_0 t}\\
			&=\sum_{k = -\infty}^{-1}\sbrak{c_k\cos\brak{2\pi kf_0 t}+jc_k\sin\brak{2\pi kf_0 t}} \\ &+c_0+\sum_{k = 1}^{\infty}\sbrak{c_k\cos\brak{2\pi kf_0 t}+jc_k\sin\brak{2\pi kf_0 t}}\\
			&=\sum_{k = 1}^{\infty}\sbrak{c_{-k}\cos\brak{2\pi kf_0 t}-jc_{-k}\sin\brak{2\pi kf_0 t}}\\ &+c_0+\sum_{k = 1}^{\infty}\sbrak{c_k\cos\brak{2\pi kf_0 t}+jc_k\sin\brak{2\pi kf_0 t}}\\
			&=c_0+\sum_{k = 1}^{\infty}\bigg(\brak{c_k+c_{-k}}\cos\brak{2\pi kf_0 t}\\ &+j\brak{c_k-c_{-k}}\sin\brak{2\pi kf_0 t}\bigg)
		\end{align}
		Substituting $a_k=c_{k} + c_{-k}$ and $b_k=j(c_{k}-c_{-k})$,we get,
		\begin{align}
			x(t)&=c_0+    \sum_{k = 1}^{\infty}\brak{a_k\cos{2\pi kf_0 t}+b_k\sin{2\pi kf_0 t}}\\
			&=\sum_{k = 0}^{\infty}\brak{a_k\cos{2\pi kf_0 t}+b_k\sin{2\pi kf_0 t}}
		\end{align}
		\begin{align}
			\label{eq:u1}
			\therefore a_k&=
			\begin{cases}
				c_k+c_{-k}&k\neq0
				\\
				c_0&k=0
			\end{cases}\\
			\label{eq:u2}
			b_k&=j\brak{c_k-c_{-k}}
		\end{align}
		Using \eqref{eq:one-Z},
		\begin{align}
			c_k &= f_0\int_{-\frac{1}{2f_0}}^{\frac{1}{2f_0}}x(t)e^{-j2\pi kf_0 t}\, dt\\
			c_{-k} &= f_0\int_{-\frac{1}{2f_0}}^{\frac{1}{2f_0}}x(t)e^{j2\pi kf_0 t}\, dt\end{align}
		\begin{align}
			a_k=c_k+c_{-k}&= f_0\int_{-\frac{1}{2f_0}}^{\frac{1}{2f_0}}x(t)\sbrak{e^{-j2\pi kf_0 t}+e^{j2\pi kf_0 t}}\, dt\\
			&=2f_0\int_{-\frac{1}{2f_0}}^{\frac{1}{2f_0}}x(t)\cos\brak{2\pi kf_0t}\, dt
		\end{align}
		Similarly, for $b_k$, we get,
		\begin{align}
			b_k=-j\cbrak{2f_0\int_{-\frac{1}{2f_0}}^{\frac{1}{2f_0}}x(t)\sin\cbrak{2\pi kf_0t}\, dt}
		\end{align}
		\item Find $a_k$ and $b_k$ for 
		\eqref{eq:x(t)}\\
		\solution
		Using \eqref{eq:u1} and \eqref{eq:u2} with \eqref{eq:ck},
		\begin{align}
			a_k&=c_k+c_{-k}=\begin{cases}
				\frac{4A_0}{\pi\brak{1-k^2}}&k=even
				\\
				\frac{2A_0}{\pi}&k=0
				\\
				0&k=odd
			\end{cases}\\
			b_k&=j\brak{c_k-c_{-k}}=0
		\end{align}
		\item Verify 
		\eqref{eq:one-Z-real}
		using python.\\
		\solution 
		\begin{lstlisting}
			wget https://github.com/Pradeep8802/EE3900-Digital-Signal-Processing/blob/main/charger/codes/2.6.py
			python3 2.3.py
		\end{lstlisting}
		\begin{figure}[!ht]
			\centering
			\includegraphics[width=\columnwidth]{./figs/2_6.png}
			\caption{}
			%\label{fig:ckt}
		\end{figure}
	\end{enumerate}
	\section{Fourier Transform}
	\begin{enumerate}[label=\thesection.\arabic*
		,ref=\thesection.\theenumi]
		\item 
		\begin{align}
			\delta(t)&=0, \quad t\neq 0 \\
			\int_{-\infty}^{\infty}\delta(t) \, dt&= 1
		\end{align}
		\item The Fourier Transform of $g(t)$ is
		\begin{align}
			G(f)=\int_{-\infty}^{\infty}g(t)e^{-j2\pi ft}\,dt
			\label{eq:fourier}
		\end{align}
		\item Show that 
		\begin{align}
			g(t-t_0)&\system{F}G(f)e^{-j2\pi ft_0}
			\label{eq:t-shift}
		\end{align}
		\solution Let us consider $x=t-t_0$. Fourier transform of $g(t-t_0)$ is given as
		\begin{align}
			g(t-t_0)&\system{F}\int_{-\infty}^{\infty}
			g(t-t_0)e^{-\j2\pi ft}\,dt \\
			&=\int_{-\infty}^{\infty}
			g(t-t_0)e^{-\j2\pi f((t-t_0) + t_0)}\,du \\
			&=\int_{-\infty}^{\infty}
			g(x)e^{-j2\pi f(x+t_0)}\,dt\\
			&=\int_{-\infty}^{\infty}
			g(x)e^{-j2\pi f(x+t_0)}\,d(x-t_0)\\
			&=\int_{-\infty}^{\infty}
			e^{-\j2\pi f t_0}g(x)e^{-j2\pi fx}\,d(x)\\
			\label{eq:3.3}
			&=e^{-\j2\pi f t_0} \cbrak{\int_{-\infty}^{\infty}
				g(x)e^{-\j2\pi fx}\,d(x)}
		\end{align}
		Using \eqref{eq:fourier} in equation \eqref{eq:3.3}, we get,
		\begin{align}
			\label{eq:3.3.1}
			g(t-t_0)\system{F}G(f)e^{-\j2\pi ft_0}
		\end{align}
		\item Show that
		\begin{align}
			\label{eq:inverse}
			G(t)&\system{F}g(-f)
		\end{align}
		\solution 
		Let $g(t)\system{F}G(f)$ , then 
		\begin{align}
			\label{eq:inverseFourier}
			g(t)=\int_{-\infty}^{\infty}G(f)e^{\j2\pi ft}\,df
		\end{align}
		Consider $g(-k)$,
		\begin{align}
			g(-k)=\int_{-\infty}^{\infty}G(f)e^{\j2\pi fk}\,df
		\end{align}
		Let $f=t$,then,
		\begin{align}
			\label{eq:3.4}
			g(-k)=\int_{-\infty}^{\infty}G(t)e^{\j2\pi tk}\,dt
		\end{align}
		Substituting $k=f$ and in the \eqref{eq:3.4}, we get, 
		\begin{align}
			\label{eq:3.4.1}
			g(-f)=\int_{-\infty}^{\infty}G(t)e^{\j2\pi ft}\,dt
		\end{align}
		Comparing \eqref{eq:3.4.1} with \eqref{eq:fourier}, we can say that,
		\begin{align}
			G(t)&\system{F}g(-f)
		\end{align}
		%	\begin{align}
			%		g(t)=\int_{-\infty}^{\infty}G(f)e^{\j2\pi ft}\,df
			%		\label{eq:duality}
			%	\end{align}
		%	Hence, setting $t := -f$ and $f := t$, which implies $df = dt$,
		%	\begin{align}
			%		g(-f)&=\int_{-\infty}^{\infty}G(t)e^{-\j2\pi ft}\,dt \\
			%		\implies G(t)&\system{F}g(-f)
			%	\end{align}
		\item $\delta(t)\system{F}?$\\
		\solution From \eqref{eq:fourier}, fourier transform of $\delta(t)$ is,
		\begin{align}
			\delta(t)&\system{F} \int_{-\infty}^{\infty}\delta(t) e^{-\j2\pi ft} \,dt\\
			&=\int_{-\infty}^{\infty}\delta(0) e^{-\j2\pi f 0} \,dt\\
			&=\int_{-\infty}^{\infty}\delta(0) \,dt \\
			&=1
		\end{align}
		Hence,
		$\delta(t) \system{F} 1$
		%	We have, from the definition of $\delta(t)$,
		%	\begin{align}
			%		\delta(t)&\system{F}\int_{-\infty}^{\infty}\delta(t)e^{-\j2\pi ft}\, dt \\
			%		&=\int_{-\infty}^{\infty}\delta(0)\, dt \\
			%		&=\int_{-\infty}^{\infty}\delta(t)\, dt = 1
			%		\label{eq:fourier-delta}
			%	\end{align}
		\item $e^{-j2\pi f_0t}\system{F}?$\\
		\solution Suppose $g(t)\system{F}G(f)$. Hence,
		\begin{align}
			g(t)&\system{F} \int_{-\infty}^{\infty}g(t)e^{-\j2\pi ft}\\
			g(t)e^{-j2\pi f_0t}&\system{F} \int_{-\infty}^{\infty}g(t)e^{-\j2\pi ft}e^{-j2\pi f_0t}\\
			g(t)e^{-j2\pi f_0t}&\system{F} G(f)e^{-j2\pi f_0 f}\\
		\end{align}
		From \eqref{eq:3.4.1},
		\begin{align}
			g(t-f_0)&\system{F}G(f)e^{-\j2\pi ft f_0}\\
			g(t)e^{-j2\pi f_0t}&\system{F} G(f)e^{-j2\pi f_0 f}\\
		\end{align}
		From \eqref{eq:inverseFourier},
		\begin{align}
			\delta(t)&\system{F}1\\
			1&\system{F}\delta(-f)=\delta(f)
		\end{align}
		Hence, 
		\begin{align}
			g(t-f_0)&\system{F} \delta((f+f_0))
		\end{align}
		\begin{align}
			g(t)e^{\j2\pi f_0t}&\system{F}\int_{-\infty}^{\infty}
			g(t)e^{-\j2\pi\brak{f-f_0}t}\, dt \\
			&=G(f-f_0)
			\label{eq:f-shift}
		\end{align}
		%Using \eqref{eq:duality} in \eqref{eq:fourier-delta}, $1\system{F}\delta(-f)$.
		%Hence, applying \eqref{eq:f-shift},
		Hence,
		\begin{align}
			e^{-\j2\pi f_0t}\system{F}\delta(-(f+f_0)) = \delta(f+f_0)
			\label{eq:fourier-exp}
		\end{align}
		\item $\cos(2\pi f_0t)\system{F}?$\\
		\solution We know that 
		\begin{align}
			\cos\brak{2\pi f_0t} = \frac{1}{2}
			\brak{e^{\j2\pi f_0t} + e^{-\j2\pi f_0t}} \\
		\end{align}
		Hence,
		\begin{align}
			\mathcal{F}(\cos\brak{2\pi f_0t})&=\mathcal{F}(\frac{1}{2}
			\brak{e^{\j2\pi f_0t} + e^{-\j2\pi f_0t}})\\
			\mathcal{F}(\cos\brak{2\pi f_0t})&=\frac{1}{2} \mathcal{F}(
			\brak{e^{\j2\pi f_0t}})+ \frac{1}{2} \mathcal{F}(e^{-\j2\pi f_0t})\\
			&=\frac{1}{2} \mathcal{F}(
			\brak{e^{\j2\pi f_0t}})+ \frac{1}{2} \mathcal{F}(e^{-\j2\pi f_0t})\\
			&=\frac{1}{2}\brak{\delta\brak{f-f_0} + \delta\brak{f+f_0}}
		\end{align}
		%
		%	Using the linearity of the Fourier 
		%	Transform and \eqref{fourier-exp},
		%	\begin{align}
			%		\cos\brak{2\pi f_0t} &= \frac{1}{2}
			%		\brak{e^{\j2\pi f_0t} + e^{-\j2\pi f_0t}} \\
			%		&\system{F}\frac{1}{2}\brak{\delta\brak{f+f_0} + \delta\brak{f-f_0}}
			%	\end{align}
		\item Find the Fourier Transform of $x(t)$ and plot it. Verify using python.\\
		\solution As obtained earlier, from equation \eqref{eq:ck},
		\begin{align}
			x(t)=\sum_{k=-\infty}^{\infty}c_k e^{j2\pi k f_0 t}\\
			e^{j 2 \pi k f_0 t}\system{F}=\delta{f-kf_0}
		\end{align}
		Hence, from the value of $c_k$,
		\begin{align}
			x(t)&\system{F}\sum_{k=-\infty}^{\infty} \delta\brak{f+kf_0} c_k\\
			\implies x(t)&\system{F}\sum_{k=-\infty}^{\infty} \frac{2A_0}{\pi} \frac{\delta\brak{f+2kf_0}}{1-4k^2} 
		\end{align} 
		\begin{figure}[!ht]
			\includegraphics[width=\columnwidth]{figs/3_8.png}
			\caption{Fourier Transform of $x(t)$.}
			\label{fig:fourier-xt}
		\end{figure}
		Fourier transform of $x(t)$ is verified in the following figure. 
		The figure is plotted using the below python code.
		\begin{lstlisting}
			wget https://github.com/Pradeep8802/EE3900-Digital-Signal-Processing/blob/main/charger/codes/3.8.py
		\end{lstlisting} 
		%	
		\item Show that
		\begin{align}
			\rect{t} \system{F} \sinc{f}
		\end{align}
		Verify using python.\\
		\solution We know that,
		\begin{align}
			rect(t)&=
			\begin{cases}
				0 \quad t<\frac{-1}{2}\\
				1 \quad \frac{-1}{2}<t<\frac{1}{2}\\
				0 \quad t>\frac{1}{2}
			\end{cases}\\
			\sinc(f)&=\frac{\sin{\pi f}}{\pi f}
		\end{align}
		Applying fourier transform we get,
		\begin{align}
			\rect{t}&\system{F}\int_{-\infty}^{\infty}\rect{t} e^{-\j2\pi ft}\, dt \\
			&=\int_{-\frac{1}{2}}^{\frac{1}{2}}e^{-\j 2\pi ft}\, dt \\
			\label{eq:fourierrect}
			&=\frac{e^{\j\pi f} - e^{-\j\pi f}}{\j2\pi f} = \frac{\sin{\pi f}}{\pi f} = \sinc{f}
		\end{align}
		\begin{figure}[!ht]
			\includegraphics[width=\columnwidth]{figs/3_9.png}
			\caption{Fourier Transform of $\rect{t})$.}
			\label{fig:3.9}
		\end{figure}
		The below python code plots the figure \ref{fig:3.9}
		\begin{lstlisting}
			wget https://github.com/Pradeep8802/EE3900-Digital-Signal-Processing/blob/main/charger/codes/3.9.py
		\end{lstlisting} 
		\item $\sinc{t}\system{F} ?$  Verify using python.\\
		\solution From \eqref{eq:3.3}, we have 
		
		\begin{align}
			\sinc{t}\system{F}\rect(-f)=\rect{f}
			\label{eq:fourier-sinc}
		\end{align}
		\begin{figure}[!ht]
			\includegraphics[width=\columnwidth]{figs/3_10.png}
			\caption{Fourier Transform of $\rect{t})$.}
			\label{fig:3.10}
		\end{figure}
		Since $\rect{f}$ is an even function.
		The below python code plots the figure \ref{fig:3.10}
		\begin{lstlisting}
			wget https://github.com/Pradeep8802/EE3900-Digital-Signal-Processing/blob/main/charger/codes/3.10.py
		\end{lstlisting} 
		%	The python code \texttt{codes/3\_10.py} verifies \eqref{eq:fourier-sinc}
		%	by plotting Fig. \ref{fig:fourier-sinc}.
		%	\begin{figure}[!ht]
			%		\includegraphics[width=\columnwidth]{figs/3_10.png}
			%		\caption{Fourier Transform of $\sinc(t)$.}
			%		\label{fig:fourier-sinc}
			%	\end{figure}
	\end{enumerate}