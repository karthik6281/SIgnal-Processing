\documentclass[journal,12pt,twocolumn]{IEEEtran}
%
\usepackage{setspace}
\usepackage{gensymb}
\usepackage{xcolor}
\usepackage{caption}
%\usepackage{subcaption}
%\doublespacing
\singlespacing

%\usepackage{graphicx}
%\usepackage{amssymb}
%\usepackage{relsize}
\usepackage[cmex10]{amsmath}
\usepackage{mathtools}
%\usepackage{amsthm}
%\interdisplaylinepenalty=2500
%\savesymbol{iint}
%\usepackage{txfonts}
%\restoresymbol{TXF}{iint}
%\usepackage{wasysym}
\usepackage{hyperref}
\usepackage{amsthm}
\usepackage{mathrsfs}
\usepackage{txfonts}
\usepackage{stfloats}
\usepackage{cite}
\usepackage{cases}
\usepackage{subfig}
%\usepackage{xtab}
\usepackage{longtable}
\usepackage{multirow}
%\usepackage{algorithm}
%\usepackage{algpseudocode}
%\usepackage{enumerate}
\usepackage{enumitem}
\usepackage{mathtools}
%\usepackage{iithtlc}
%\usepackage[framemethod=tikz]{mdframed}
\usepackage{listings}


%\usepackage{stmaryrd}


%\usepackage{wasysym}
%\newcounter{MYtempeqncnt}
\DeclareMathOperator*{\Res}{Res}
%\renewcommand{\baselinestretch}{2}
\renewcommand\thesection{\arabic{section}}
\renewcommand\thesubsection{\thesection.\arabic{subsection}}
\renewcommand\thesubsubsection{\thesubsection.\arabic{subsubsection}}

\renewcommand\thesectiondis{\arabic{section}}
\renewcommand\thesubsectiondis{\thesectiondis.\arabic{subsection}}
\renewcommand\thesubsubsectiondis{\thesubsectiondis.\arabic{subsubsection}}

%\renewcommand{\labelenumi}{\textbf{\theenumi}}
%\renewcommand{\theenumi}{P.\arabic{enumi}}

% correct bad hyphenation here
\hyphenation{op-tical net-works semi-conduc-tor}

\lstset{
language=Python,
frame=single, 
breaklines=true,
columns=fullflexible
}



\begin{document}
%

\theoremstyle{definition}
\newtheorem{theorem}{Theorem}[section]
\newtheorem{problem}{Problem}
\newtheorem{proposition}{Proposition}[section]
\newtheorem{lemma}{Lemma}[section]
\newtheorem{corollary}[theorem]{Corollary}
\newtheorem{example}{Example}[section]
\newtheorem{definition}{Definition}[section]
%\newtheorem{algorithm}{Algorithm}[section]
%\newtheorem{cor}{Corollary}
\newcommand{\BEQA}{\begin{eqnarray}}
\newcommand{\EEQA}{\end{eqnarray}}
\newcommand{\define}{\stackrel{\triangle}{=}}
\newcommand{\myvec}[1]{\ensuremath{\begin{pmatrix}#1\end{pmatrix}}}
\bibliographystyle{IEEEtran}
%\bibliographystyle{ieeetr}
\providecommand{\nCr}[2]{\,^{#1}C_{#2}} % nCr
\providecommand{\nPr}[2]{\,^{#1}P_{#2}} % nPr
\providecommand{\mbf}{\mathbf}
\providecommand{\pr}[1]{\ensuremath{\Pr\left(#1\right)}}
\providecommand{\qfunc}[1]{\ensuremath{Q\left(#1\right)}}
\providecommand{\sbrak}[1]{\ensuremath{{}\left[#1\right]}}
\providecommand{\lsbrak}[1]{\ensuremath{{}\left[#1\right.}}
\providecommand{\rsbrak}[1]{\ensuremath{{}\left.#1\right]}}
\providecommand{\brak}[1]{\ensuremath{\left(#1\right)}}
\providecommand{\lbrak}[1]{\ensuremath{\left(#1\right.}}
\providecommand{\rbrak}[1]{\ensuremath{\left.#1\right)}}
\providecommand{\cbrak}[1]{\ensuremath{\left\{#1\right\}}}
\providecommand{\lcbrak}[1]{\ensuremath{\left\{#1\right.}}
\providecommand{\rcbrak}[1]{\ensuremath{\left.#1\right\}}}
\theoremstyle{remark}
\newtheorem{rem}{Remark}
\newcommand{\sgn}{\mathop{\mathrm{sgn}}}
\providecommand{\abs}[1]{\left\vert#1\right\vert}
\providecommand{\res}[1]{\Res\displaylimits_{#1}} 
\providecommand{\norm}[1]{\lVert#1\rVert}
\providecommand{\mtx}[1]{\mathbf{#1}}
\providecommand{\mean}[1]{E\left[ #1 \right]}
\providecommand{\fourier}{\overset{\mathcal{F}}{ \rightleftharpoons}}
\providecommand{\ztrans}{\overset{\mathcal{Z}}{ \rightleftharpoons}}
%\providecommand{\hilbert}{\overset{\mathcal{H}}{ \rightleftharpoons}}
\providecommand{\system}{\overset{\mathcal{H}}{ \longleftrightarrow}}
	%\newcommand{\solution}[2]{\textbf{Solution:}{#1}}
\let\vec\mathbf
\newcommand{\solution}{\noindent \textbf{Solution: }}
\providecommand{\dec}[2]{\ensuremath{\overset{#1}{\underset{#2}{\gtrless}}}}
\numberwithin{equation}{section}
%\numberwithin{equation}{subsection}
%\numberwithin{problem}{subsection}
%\numberwithin{definition}{subsection}
\makeatletter
\@addtoreset{figure}{problem}
\makeatother
\let\StandardTheFigure\thefigure
%\renewcommand{\thefigure}{\theproblem.\arabic{figure}}
\renewcommand{\thefigure}{\theproblem}
%\numberwithin{figure}{subsection}
\def\putbox#1#2#3{\makebox[0in][l]{\makebox[#1][l]{}\raisebox{\baselineskip}[0in][0in]{\raisebox{#2}[0in][0in]{#3}}}}
     \def\rightbox#1{\makebox[0in][r]{#1}}
     \def\centbox#1{\makebox[0in]{#1}}
     \def\topbox#1{\raisebox{-\baselineskip}[0in][0in]{#1}}
     \def\midbox#1{\raisebox{-0.5\baselineskip}[0in][0in]{#1}}
\vspace{3cm}


\vspace{3cm}

\title{ 
%\logo{
Digital Signal Processing
%}
%	\logo{Octave for Math Computing }
}
%\title{
%	\logo{Matrix Analysis through Octave}{\begin{center}\includegraphics[scale=.24]{tlc}\end{center}}{}{HAMDSP}
%}


% paper title
% can use linebreaks \\ within to get better formatting as desired
%\title{Matrix Analysis through Octave}
%
%
% author names and IEEE memberships
% note positions of commas and nonbreaking spaces ( ~ ) LaTeX will not break
% a structure at a ~ so this keeps an author's name from being broken across
% two lines.
% use \thanks{} to gain access to the first footnote area
% a separate \thanks must be used for each paragraph as LaTeX2e's \thanks
% was not built to handle multiple paragraphs
%

\author{Ravula Karthik% <-this % stops a space
%\thanks{J. Doe and J. Doe are with Anonymous University.}% <-this % stops a space
%\thanks{Manuscript received April 19, 2005; revised January 11, 2007.}}
}
% note the % following the last \IEEEmembership and also \thanks - 
% these prevent an unwanted space from occurring between the last author name
% and the end of the author line. i.e., if you had this:
% 
% \author{....lastname \thanks{...} \thanks{...} }
%                     ^------------^------------^----Do not want these spaces!
%
% a space would be appended to the last name and could cause every name on that
% line to be shifted left slightly. This is one of those "LaTeX things". For
% instance, "\textbf{A} \textbf{B}" will typeset as "A B" not "AB". To get
% "AB" then you have to do: "\textbf{A}\textbf{B}"
% \thanks is no different in this regard, so shield the last } of each \thanks
% that ends a line with a % and do not let a space in before the next \thanks.
% Spaces after \IEEEmembership other than the last one are OK (and needed) as
% you are supposed to have spaces between the names. For what it is worth,
% this is a minor point as most people would not even notice if the said evil
% space somehow managed to creep in.



% The paper headers
%\markboth{Journal of \LaTeX\ Class Files,~Vol.~6, No.~1, January~2007}%
%{Shell \MakeLowercase{\textit{et al.}}: Bare Demo of IEEEtran.cls for Journals}
% The only time the second header will appear is for the odd numbered pages
% after the title page when using the twoside option.
% 
% *** Note that you probably will NOT want to include the author's ***
% *** name in the headers of peer review papers.                   ***
% You can use \ifCLASSOPTIONpeerreview for conditional compilation here if
% you desire.




% If you want to put a publisher's ID mark on the page you can do it like
% this:
%\IEEEpubid{0000--0000/00\$00.00~\copyright~2007 IEEE}
% Remember, if you use this you must call \IEEEpubidadjcol in the second
% column for its text to clear the IEEEpubid mark.



% make the title area
\maketitle

%\newpage

\tableofcontents

%\renewcommand{\thefigure}{\thesection.\theenumi}
%\renewcommand{\thetable}{\thesection.\theenumi}

\renewcommand{\thefigure}{\theenumi}
\renewcommand{\thetable}{\theenumi}

%\renewcommand{\theequation}{\thesection}


\bigskip

\begin{abstract}
This manual provides a simple introduction to digital signal processing.
\end{abstract}
\section{Software Installation}
Run the following commands
\begin{lstlisting}
sudo apt-get update
sudo apt-get install libffi-dev libsndfile1 python3-scipy  python3-numpy python3-matplotlib 
sudo pip install cffi pysoundfile 
\end{lstlisting}
\section{Digital Filter}
\begin{enumerate}[label=\thesection.\arabic*
,ref=\thesection.\theenumi]
\item
\label{prob:input}
Download the sound file from  
\begin{lstlisting}
wget https://github.com/karthik6281/Signal-Processing/blob/main/sig-pro/codes/Sound_Noise.wav
\end{lstlisting}
%\href{http://tlc.iith.ac.in/img/sound/Sound_Noise.wav}{\url{http://tlc.iith.ac.in/img/sound/Sound_Noise.wav}}  
%in the link given below.
%\linebreak
\item
\label{prob:spectrogram}
You will find a spectrogram at \href{https://academo.org/demos/spectrum-analyzer}{\url{https://academo.org/demos/spectrum-analyzer}}. 
%\end{problem}
%%
%
%%\onecolumn
%%\input{./figs/fir}
%\begin{problem}
Upload the sound file that you downloaded in Problem \ref{prob:input} in the spectrogram  and play.  Observe the spectrogram. What do you find?
\\
%
\solution There are a lot of yellow lines between 440 Hz to 5.1 KHz.  These represent the synthesizer key tones. Also, the key strokes
are audible along with background noise.
% By observing spectrogram, it clearly shows that tonal frequency is under 4kHz. And above 4kHz only noise is present.
\item
\label{prob:output}
Write the python code for removal of out of band noise and execute the code.
\\
\solution
\lstinputlisting{./codes/Cancel_noise.py}
%\begin{figure}[h]
%\centering
%\includegraphics[width=\columnwidth]{enc_block_diag.png}
%\caption{}
%\label{fig:convolution encoder}
%\end{figure}
%\input{block_enc}
\item
The output of the python script in Problem \ref{prob:output} is the audio file Sound\_With\_ReducedNoise.wav. Play the file in the spectrogram in Problem \ref{prob:spectrogram}. What do you observe?
\\
\solution The key strokes as well as background noise is subdued in the audio.  Also,  the signal is blank for frequencies above 5.1 kHz.

\end{enumerate}
\section{Difference Equation}
\begin{enumerate}[label=\thesection.\arabic*,ref=\thesection.\theenumi]
\item Let
\begin{equation}
x(n) = \cbrak{\underset{\uparrow}{1},2,3,4,2,1} \label{x(n)}
\end{equation}
Sketch $x(n)$.

\solution The following code yields Fig. \ref{fig:3.1}.
\begin{lstlisting}
wget https://github.com/karthik6281/Signal-Processing/blob/main/sig-pro/codes/3_1.py
\end{lstlisting} 
	\begin{figure}[!ht]
	\begin{center}
		\includegraphics[width=\columnwidth]{./figs/3_1.png}
	\end{center}
	\captionof{figure}{}
	\label{fig:3.1}	
\end{figure}
\item Let
\begin{multline}
\label{eq:iir_filter}
y(n) + \frac{1}{2}y(n-1) = x(n) + x(n-2), 
\\
 y(n) = 0, n < 0
\end{multline}
Sketch $y(n)$.
\\
\solution The following code yields Fig. \ref{fig:3.2}.
\begin{lstlisting}
wget https://github.com/karthik6281/Signal-Processing/blob/main/sig-pro/codes/3_2.py
\end{lstlisting}
\begin{figure}[!ht]
\begin{center}
\includegraphics[width=\columnwidth]{./figs/3_2.png}
\end{center}
\captionof{figure}{}
\label{fig:3.2}	
\end{figure}

\item Repeat the above exercise using a C code.
\solution The c code can be obtained from
\begin{lstlisting}
wget https://github.com/karthik6281/Signal-Processing/blob/main/sig-pro/codes/3_3.c
\end{lstlisting}
\end{enumerate}

\section{$Z$-transform}
\begin{enumerate}[label=\thesection.\arabic*]
	\item The $Z$-transform of $x(n)$ is defined as
	%
	\begin{equation}
		\label{eq:z_trans}
		X(z)={\mathcal {Z}}\{x(n)\}=\sum _{n=-\infty }^{\infty }x(n)z^{-n}
	\end{equation}
	%
	Show that
	\begin{equation}
		\label{eq:shift1}
		{\mathcal {Z}}\{x(n-1)\} = z^{-1}X(z)
	\end{equation}
	and find
	\begin{equation}
		{\mathcal {Z}}\{x(n-k)\} 
	\end{equation}
	\solution From \eqref{eq:z_trans},
	\begin{align}
		{\mathcal {Z}}\{x(n-1)\} &=\sum _{n=-\infty }^{\infty }x(n-1)z^{-n}
		\\
		&=\sum _{n=-\infty }^{\infty }x(n)z^{-n-1} = z^{-1}\sum _{n=-\infty }^{\infty }x(n)z^{-n}
	\end{align}
	resulting in \eqref{eq:shift1}. Similarly, it can be shown that
	%
	\begin{equation}
		\label{eq:z_trans_shift}
		{\mathcal {Z}}\{x(n-k)\} = z^{-k}X(z)
	\end{equation}
	\item Obtain $X(z)$ for $x(n)$ defined in problem 
	\ref{def:xn}.
	\solution $Z$-transform of x(n),$X(z)$ is given by
	\begin{align}
		\mathcal{Z}\{x(n)\}&=\sum_{n=-\infty}^\infty x(n)z^{-n}\\
		&=\sum_{n=0}^5x(n)z^{-n}\\
		&=1+2z^{-1}+3z^{-2}+4z^{-3}+2z^{-4}+z^{-5}
	\end{align}
	%
	%For $x(n) = \cbrak{\underset{\uparrow}{1},2,3,4,2,1}$
	%\begin{align}
	%	{\mathcal {Z}}\{x(n)\} &=\sum _{n=-\infty }^{\infty }x(n)z^{-n}
	%	\\
	%	{\mathcal {Z}}\{x(n)\} &=x(0)
	%	&=\sum _{n=-\infty }^{\infty }x(n)z^{-n-1} = z^{-1}\sum _{n=-\infty }^{\infty }x(n)z^{-n}
	%\end{align}
	%\begin{equation}
	%	x(n) = \cbrak{\underset{\uparrow}{1},2,3,4,2,1}
	%\end{equation}
	%
	%
	\item Find
	%
	\begin{equation}
		H(z) = \frac{Y(z)}{X(z)}
	\end{equation}
	%
	from  \eqref{eq:iir_filter} assuming that the $Z$-transform is a linear operation.
	\\
	\solution  Applying \eqref{eq:z_trans_shift} in \eqref{eq:iir_filter},
	\begin{align}
		Y(z) + \frac{1}{2}z^{-1}Y(z) &= X(z)+z^{-2}X(z)
		\\
		\implies \frac{Y(z)}{X(z)} &= \frac{1 + z^{-2}}{1 + \frac{1}{2}z^{-1}}
		\label{eq:freq_resp}
	\end{align}
	%
	\item Find the Z transform of 
	\begin{equation}
		\delta(n)
		=
		\begin{cases}
			1 & n = 0
			\\
			0 & \text{otherwise}
		\end{cases}
	\end{equation}
	and show that the $Z$-transform of
	\begin{equation}
		\label{eq:unit_step}
		u(n)
		=
		\begin{cases}
			1 & n \ge 0
			\\
			0 & \text{otherwise}
		\end{cases}
	\end{equation}
	is
	\begin{equation}
		U(z) = \frac{1}{1-z^{-1}}, \quad \abs{z} > 1
	\end{equation}
	\solution The $Z$-transform of $\delta(n)$ is defined as
	\begin{align}
		{\mathcal {Z}}\{\delta(n)\}&=\sum _{n=-\infty }^{\infty }\delta(n)z^{-n}\\
		&=\delta(0)z^{-0}\\
		&=1
	\end{align}
	Hence we can say that
	\begin{equation}
		\label{eq:z_trans1}
		\delta(n) \ztrans 1
	\end{equation}
	and from \eqref{eq:unit_step},
	\begin{align}
		U(z) &= \sum _{n= 0}^{\infty}z^{-n}
		\\
		&=\frac{1}{1-z^{-1}}, \quad \abs{z} > 1
	\end{align}
	using the fomula for the sum of an infinite geometric progression.
	%
	\item Show that 
	\begin{equation}
		\label{eq:anun}
		a^nu(n) \ztrans \frac{1}{1-az^{-1}} \quad \abs{z} > \abs{a}
	\end{equation}
	\solution
	\begin{align}
		{\mathcal {Z}}\{a^nu(n)\}&=\sum _{n=-\infty }^{\infty }a^nu(n)z^{-n}\\
		&=\sum _{n=0 }^{\infty }a^nz^{-n}\\
		&=\sum _{n=0 }^{\infty }(z^{-1}a)^{n}\\
		&=\frac{1}{1-az^{-1}}, \quad \abs{z^{-1}a} < 1\\
		&=\frac{1}{1-az^{-1}}, \quad \abs{z} > \abs{a} 
	\end{align}
	using the fomula for the sum of an infinite geometric progression.
	%
	\item 
	Let
	\begin{equation}
		H\brak{e^{\j \omega}} = H\brak{z = e^{\j \omega}}.
	\end{equation}
	Plot $\abs{H\brak{e^{\j \omega}}}$.  Is it periodic? If so, find the period.  $H(e^{\j \omega})$ is
	known as the {\em Discret Time Fourier Transform} (DTFT) of $h(n)$.
	\\
	\solution $H(e^{jw})$ is given by
	\begin{align}
		H(e^{jw})&=\frac{1+(e^{jw})^{-2}}{1+\frac{1}{2}(e^{jw})^{-1}}\\
		&=2\frac{1+\cos(-2\omega)+j\sin(-2\omega)}{2+\cos(-\omega)+j\sin(-\omega)}\\
		&=2\frac{1+\cos(2\omega)-j\sin(2\omega)}{2+\cos(\omega)-j\sin(\omega)}\\
		&=2\frac{2\cos^2(\omega)-2j\sin(\omega)\cos(\omega)}{2+\cos(\omega)-j\sin(\omega)}\\
		&=4\cos(\omega)\frac{\cos(\omega)-j\sin(\omega)}{2+\cos(\omega)-j\sin(\omega)}\\
		&=4|\cos(\omega)| \frac{e^{jw}}{2+e^{jw}}
	\end{align}
	So,
	\begin{align}
		|H(e^{jw})|&=4 |\cos(\omega)| \frac{|e^{jw}|}{|2+e^{jw}|}\\
		&=\frac{|4\cos(\omega)|}{5+4\cos(\omega)}
	\end{align}
	 $\abs{H\brak{e^{\j\omega}}}$ is periodic with period $\pi$.( The LCM of the period of $|\cos(\omega)|$ and $5+4\cos(\omega)$ is $2\pi$)
	The graph of $\abs{H\brak{e^{\j\omega}}}$ is symmetric with respect to y-axis. It is continuous over $\omega$. The following code plots Fig. \ref{fig:4.5}.
	\begin{lstlisting}
		wget https://github.com/karthik6281/Signal-Processing/blob/main/Assignment1/codes/4_6.py
	\end{lstlisting}
	\begin{figure}[!ht]
		\centering
		\includegraphics[width=\columnwidth]{./figs/4_6.png}
		\caption{$\abs{H\brak{e^{\j\omega}}}$}
		\label{fig:4.5}
	\end{figure}
	\item Express $h(n)$ in terms of $H\brak{e^{\j \omega}}$.
	
	\solution 
	\begin{align}
		&\int_{-\pi}^\pi H(e^{j\omega})e^{j\omega k }d\omega = \sum_{-\infty}^{\infty} h(n) \int_{-\pi}^{\pi} e^{j\omega n }e^{-j\omega k } d\omega \\ 
		&= \sum_{-\infty}^{\infty} h(n)\int_{-\pi}^{\pi}(\cos({n-k}) + i\sin({n-k}))d\omega
	\end{align}
	\begin{align}
	\int_{-\pi}^{\pi}(\cos({n-k}) + i\sin({n-k}))d\omega = \begin{cases}
			2\pi & n = k
			\\
			0 & n \neq k
		\end{cases} 
	\end{align}
	\begin{align}
	 \therefore h(n) = \frac{\int_{-\pi}^\pi H(e^{j\omega})e^{j\omega n }d\omega}{2\pi}
	\end{align}
\end{enumerate}
\section{Impulse Response}
\begin{enumerate}[label=\thesection.\arabic*]
	\item Using long division, 
	find
	\begin{align}
		h(n), \quad n < 5
	\end{align}
	for H(z) in 
	\eqref{eq:freq_resp}.
	
	\solution $H(z)$ is given by
	\begin{align}
		H(z)=\frac{1+z^{-2}}{1+\frac{1}{2}z^{-1}}=\frac{2+2z^{-2}}{2+z^{-1}}
	\end{align}
	\begin{align}
		&2z^{-1}-4 \nonumber \\	
		z^{-1}+2\hspace{2mm}&\overline{\big)\hspace{2mm} 2z^{-2}+2\hspace{15mm}} \nonumber\\
		& \hspace{2mm} 2z^{-2}+4z^{-1} \nonumber\\
		&\overline{\hspace{11mm}-4z^{-1}+2\hspace{5mm}} \nonumber\\
		&\hspace{9mm}-4z^{-1}-8 \nonumber\\ 
		&\overline{\hspace{24mm}10} \nonumber
	\end{align}
	So,
	\begin{align}
		H(z)&=2z^{-1}-4+\frac{10}{z^{-1}+2}\\
		&=2z^{-1}-4+\frac{5}{\frac{1}{2}z^{-1}+1}\\
		&=2z^{-1}-4+5\sum_{n=0}^\infty \brak{-\frac{z^{-1}}{2}}^{n}\\
		&=1-\frac{1}{2}z^{-1}+5\sum_{n=2}^\infty \brak{-\frac{1}{2}}^{n}z^{-n}
	\end{align}
	So,h(n) will be given by 
	\begin{align}
		h(n)=\begin{cases}
			\label{eq:h_n_def}
			5\times \brak{-\frac{1}{2}}^n  & n\geq 2\\
			\brak{-\frac{1}{2}}^n  &2>n\geq0\\
			0 &n<0
		\end{cases}
	\end{align}
	\item \label{prob:impulse_resp}
	Find an expression for $h(n)$ using $H(z)$, given that 
	%in Problem \ref{eq:ztransab} and \eqref{eq:anun}, given that
	\begin{equation}
		\label{eq:impulse_resp}
		h(n) \ztrans H(z)
	\end{equation}
	and there is a one to one relationship between $h(n)$ and $H(z)$. $h(n)$ is known as the {\em impulse response} of the
	system defined by \eqref{eq:iir_filter}.
	\\
	\solution From \eqref{eq:freq_resp},
	\begin{align}
		H(z) &= \frac{1}{1 + \frac{1}{2}z^{-1}} + \frac{ z^{-2}}{1 + \frac{1}{2}z^{-1}}
		\\
		\implies h(n) &= \brak{-\frac{1}{2}}^{n}u(n) + \brak{-\frac{1}{2}}^{n-2}u(n-2)
	\end{align}
	using \eqref{eq:anun} and \eqref{eq:z_trans_shift}.
	\item Sketch $h(n)$. Is it bounded? Justify theoreti-
	cally.
	\\
	\solution The following code plots Fig. \ref{fig:5.2}.
	\begin{lstlisting}
		wget https://github.com/karthik6281/Signal-Processing/blob/main/Assignment1/codes/5_3.py
	\end{lstlisting}
	on simplfying we get h(n) as
	\begin{align}
		\begin{cases}
			%\label{eq:h_n_def}
			5\times \brak{-\frac{1}{2}}^n  & n\geq 2\\
			\brak{-\frac{1}{2}}^n  &2>n\geq0\\
			0 &n<0
		\end{cases}
	\end{align}
	\begin{align}
		\because 5\times \brak{-\frac{1}{2}}^n \to 0 \quad	\text{for} \quad n\to \infty 
	\end{align}
	So, we can conclude that h(n) is bounded.
	\begin{figure}[!ht]
		\centering
		\includegraphics[width=\columnwidth]{./figs/5_3.png}
		\caption{$h(n)$ wrt n}
		\label{fig:5.2}
	\end{figure}
	%
     \item Convergent? Justify using the ratio test.\\
\solution  A  sequence $\cbrak{x_n}$ is convergent if 
\begin{align}
	\lim_{n \rightarrow \infty}\abs{\frac{x_{n+1}}{x_n}} < 1
\end{align}
This is known as Ratio test.\\
In this case the limit will become,
\begin{align}
	\lim_{n \rightarrow \infty}\abs{\frac{h\brak{n+1}}{h\brak{n}}} &= \lim_{n \rightarrow \infty}\abs{\frac{5\brak{\frac{-1}{2}}^{n+1}}{5\brak{\frac{-1}{2}}^{n}}} \\
	&= \frac{1}{2} < 1
\end{align}
$\therefore$ $h\brak{n}$ is convergent.
	\item The system with $h(n)$ is defined to be stable if
	\begin{equation}
		\sum_{n=-\infty}^{\infty}h(n) < \infty
	\end{equation}
	Is the system defined by \eqref{eq:iir_filter} stable for the impulse response in \eqref{eq:impulse_resp}?\\
	\solution Taking $h(n)$ as defined in \eqref{eq:h_n_def} 
	Then
	\begin{align}
		\sum_{n=-\infty}^{\infty}h(n)&= +\sum_{n=-\infty}^{-1}0+\sum_{n=0}^{1} \brak{-\frac{1}{2}}^n+\sum_{n=2}^{\infty}5\times \brak{-\frac{1}{2}}^n\\
		&=\frac{4}{3}
	\end{align}
	Since the sum is finite so the system is stable for impulsive response
	\item Verify the above result using a python code.\\
	\solution The above result is verified using the below python code
		\begin{lstlisting}
wget https://github.com/karthik6281/Signal-Processing/tree/main/Assignment1/codes/5_6.py
	\end{lstlisting}
	\item 
	Compute and sketch $h(n)$ using 
	\begin{equation}
		\label{eq:iir_filter_h}
		h(n) + \frac{1}{2}h(n-1) = \delta(n) + \delta(n-2), 
	\end{equation}
	%
	This is the definition of $h(n)$.
	\\
	\solution The following code plots Fig. \ref{fig:h_n_delta}. Note that this is the same as Fig. 
	\ref{fig:3.1}. 
	%
	\begin{lstlisting}
wget https://github.com/karthik/Signal-Processing/tree/main/Assignment1/codes/5_7.py
	\end{lstlisting}
	\begin{figure}[!ht]
		\centering
		\includegraphics[width=\columnwidth]{./figs/5_7.png}
		\caption{$h(n)$ from the definition}
		\label{fig:h_n_delta}
	\end{figure}
	%
	\item Compute 
	%
	\begin{equation}
		\label{eq:convolution}
		y(n) = x(n)*h(n) = \sum_{k=-\infty}^{\infty}x(k)h(n-k)
	\end{equation}
	%
	Comment. The operation in \eqref{eq:convolution} is known as
	{\em convolution}.
	%
	\\
	\solution The following code plots Fig. \ref{fig:y_n_convo}. Note that this is the same as 
	$y(n)$ in  Fig. 
	\ref{fig:3.1}. 
	\begin{lstlisting}
wget https://github.com/karthik/Signal-Processing/tree/main/Assignment1/codes/5_8.py
	\end{lstlisting}
	\begin{figure}[!ht]
		\centering
		\includegraphics[width=\columnwidth]{./figs/5_8.png}
		\caption{$y(n)$ from the definition of convolution}
		\label{fig:y_n_convo}
	\end{figure}
	  \item Express the above convolution using a Toeplitz matrix.\\
	\solution 
	\begin{figure}
		\centering
		\includegraphics[width = \columnwidth]{figs/5_9.png}
		\caption{Convolution of $x\brak{n}$ and $h\brak{n}$ using toeplitz matrix}
		\label{5.9}
	\end{figure}
	\begin{lstlisting}
wget https://github.com/karthik6281/Signal-Processing/tree/main/Assignment1/codes/5_9.py
	\end{lstlisting}
	From $\eqref{eq:convolution}$,we express $y\brak{n}$ as
	\begin{align}
		y\brak{n} &= \sum_{k = -\infty}^{\infty}x\brak{k}h\brak{n-k}
	\end{align}
	To understand how we can use a Toeplitz matrix, we will see what we are doing in $\eqref{eq:convolution}$ 
	\begin{align}
		y\brak{0} &= x\brak{0}h\brak{0}\\
		y\brak{1} &= x\brak{0}h\brak{1} + x\brak{1}h\brak{0}\\
		y\brak{2} &= x\brak{0}h\brak{2} + x\brak{1}h\brak{1} + x\brak{2}h\brak{0}\\
		. \nonumber&\\ 
		.& \nonumber
	\end{align}
	The same thing can be written as,
	\begin{align}
		y\brak{0} &= \myvec{h\brak{0} & 0 & 0 &.\,&.\,&.0}\myvec{x\brak{0}\\x\brak{1}\\x\brak{2}\\ . \\.\\x\brak{5}}\\
		y\brak{1} &= \myvec{h\brak{1} & h\brak{0} & 0 & 0 &.\,&.\,&.0}\myvec{x\brak{0}\\x\brak{1}\\x\brak{2}\\ . \\.\\x\brak{5}}\\
		y\brak{2} &= \myvec{h\brak{2} & h\brak{1} & h\brak{0} & 0& .\,&.0}\myvec{x\brak{0}\\x\brak{1}\\x\brak{2}\\ . \\.\\x\brak{5}}\\
		. & \nonumber \\
		.& \nonumber
	\end{align}
	Using Toeplitz matrix of $h\brak{n}$ we can simplify it as,
	\begin{align}
		y\brak{n} &= \myvec{h\brak{0} & 0 & 0 &.\,&.\,&.\,0 \\
			h\brak{1} & h\brak{0} & 0 & .\,&.\,&.\,0 \\
			h\brak{2} & h\brak{1} & h\brak{0} & .\,&.\,&.\,0 \\
			&&..\\&&..\\ 0 & 0 &  0 &.\,&.\,&.\, h\brak{m-1}}\myvec{x\brak{0}\\x\brak{1}\\x\brak{2}\\ . \\.\\x\brak{5}}\label{eq:5.9}
	\end{align}
	
	Now from $\eqref{x(n)}$ we will take n 
	\begin{align}
		x\brak{n} &= \myvec{1\\2\\3\\4\\2\\1}
	\end{align}
	And from $\eqref{eq:h_n_def}$ we will take some values of n,
	\begin{align} 
		h\brak{n} &= \myvec{1 \\ -0.5 \\ 1.25 \\. \\ . }
	\end{align}
	Now using $\eqref{eq:5.9}$,
	\begin{align}
		y\brak{n} &= x\brak{n}*h\brak{n}\\
		&= \myvec{1 & 0 & 0 &.\,&.\,&.\,0 \\
			-0.5 & 1 & 0 & .\,&.\,&.\,0 \\
			1.25 & -0.5 & 1 & .\,&.\,&.\,0 \\
			&&..\\&&..\\ 0 & 0 &  0 &.\,&.\,&.\, }\myvec{x\brak{0}\\x\brak{1}\\x\brak{2}\\ . \\.\\x\brak{5}} \\
		&= \myvec{1\\1.5\\3.25\\.\\.\\.}
	\end{align}
	
	\item Show that
	\begin{equation}
		y(n) =  \sum_{k=-\infty}^{\infty}x(n-k)h(k)
	\end{equation}
	\solution From \eqref{eq:convolution}
	\begin{align}
		y(n) = \sum_{k=-\infty}^{\infty}x(k)h(n-k)
	\end{align}
	Replacing n-k with a,we get
	\begin{align}
		y(n) &= \sum_{n-a=-\infty}^{\infty}x(n-a)h(a)\\
		&=\sum_{-a=-\infty}^{\infty}x(n-a)h(a)\\
		&=\sum_{a=-\infty}^{\infty}x(n-a)h(a)
	\end{align}
\end{enumerate}
%
%%
\section{DFT and FFT}
\begin{enumerate}[label=\thesection.\arabic*]
	\item Compute
	\begin{equation}
		X(k) \define \sum _{n=0}^{N-1}x(n) e^{-\j2\pi kn/N} \quad k = 0,1,\dots, N-1
	\end{equation}
	and $H(k)$ using $h(n)$
	
	\solution The python code can be obtained from 
	\begin{lstlisting}
		wget https://github.com/karthik6281/Signal-Processing/tree/main/Assignment1/codes/6_1.py
	\end{lstlisting}
	
	Execute the following commands
	\begin{lstlisting}
		python3 6_1.py
	\end{lstlisting}
	
	\begin{figure}[!ht]
		\centering
		\includegraphics[width=\columnwidth]{./figs/6_1.png}
		\caption{Plots of the real parts of the discrete Fourier transforms of $x(n)$ and $h(n)$}
		\label{fig-6.1}	
	\end{figure}
	
	\item Compute 
	\begin{equation}
		Y(k) = X(k)H(k)
	\end{equation}
	
	\solution The python code can be obtained from 
	\begin{lstlisting}
		wget https://github.com/karthik6281/Signal-Processing/tree/main/Assignment1/codes/6_2.py
	\end{lstlisting}
	
	Execute the following commands
	\begin{lstlisting}
		python3 6_2.py
	\end{lstlisting}
	
	\begin{figure}[!ht]
		\centering
		\includegraphics[width=\columnwidth]{./figs/6_2.png}
		\caption{Plot of $Y(k)$}
		\label{fig-6.2}	
	\end{figure}
	
	\item Compute
	\begin{equation}
		y\brak{n}={\frac {1}{N}}\sum _{k=0}^{N-1}Y\brak{k}e^{\j 2\pi kn/N} \quad n = 0,1,\dots, N-1
	\end{equation}
	
	\solution The python code can be obtained from 
	\begin{lstlisting}
		wget https://github.com/karthik6281/Signal-Processing/tree/main/Assignment1/codes/6_3.py
	\end{lstlisting}
	
	Execute the following commands
	\begin{lstlisting}
		python3 6_3.py
	\end{lstlisting}
	
		\begin{figure}[!ht]
		\centering
		\begin{center}
			\includegraphics[width=\columnwidth]{./figs/6_3_1.png}
			\caption{Plot using difference equation of $Y(k)$}
			\label{fig-6.3.1}	
		\end{center}
	    \end{figure}
	
        \begin{figure}[!ht]
        	\centering
        	\includegraphics[width=\columnwidth]{./figs/6_3_2.png}
        	\caption{Plot of the inverse discrete Fourier transform of $Y(k)$}
        	\label{fig-6.3.2}	
        \end{figure}
    
	\item Repeat the previous exercise by computing $X(k), H(k)$ and $y(n)$ through FFT and 
	IFFT.
	
	\solution The python code can be obtained from 
	\begin{lstlisting}
	wget https://github.com/karthik6281/Signal-Processing/tree/main/Assignment1/codes/6_4.py
	\end{lstlisting}
	
	Execute the following commands
	\begin{lstlisting}
		python3 6_4.py
	\end{lstlisting}

    \begin{figure}[!ht]
	\centering
     \begin{center}
     	\includegraphics[width=\columnwidth]{./figs/6_3_1.png}
     \end{center}
     \caption{Plot using difference equation of $Y(k)$}
     \label{fig-6.4.1}
    \end{figure}

     \begin{figure}[!ht]
   	 \centering
   	 \includegraphics[width=\columnwidth]{./figs/6_3_2.png}
   	 \caption{Plot of the inverse discrete Fourier transform of $Y(k)$}
   	 \label{fig-6.4.2}	
     \end{figure}
	
	\begin{figure}[!ht]
		\centering
		\includegraphics[width=\columnwidth]{./figs/6_4.png}
		\caption{Plot of $y(n)$ by fast Fourier transform}
		\label{fig-6.4.3}	
	\end{figure}
 
\end{enumerate}

\section{FFT}
% \subsection{Definitions}
\begin{enumerate}[label=\arabic*.,ref=\thesection.\theenumi]
	\numberwithin{equation}{section}
	\item The DFT of $x(n)$ is given by
	\begin{align}
		X(k) \triangleq \sum_{n=0}^{N-1} x(n) e^{-j 2 \pi k n / N}, \quad k=0,1, \ldots, N-1
	\end{align}
	\item Let 
	\begin{align}
		W_{N} = e^{-j2\pi/N} \label{eq:twiddle}
	\end{align}
	Then the $N$-point ${ DFT matrix}$ is defined as 
	\begin{align}
		\vec{F}_{N} = \sbrak{W_{N}^{mn}}, \quad 0 \le m,n \le N-1 \label{def:N-point_matrix} 
	\end{align}
	where $W_{N}^{mn}$ are the elements of $\vec{F}_{N}$.
	\item Let 
	\begin{align}
		\vec{I}_4 = \myvec{\vec{e}_4^{1} &\vec{e}_4^{2} &\vec{e}_4^{3} &\vec{e}_4^{4} }
	\end{align}
	be the $4\times 4$ identity matrix.  Then the 4 point {\em DFT permutation matrix} is defined as 
	\begin{align}
		\vec{P}_4 = \myvec{\vec{e}_4^{1} &\vec{e}_4^{3} &\vec{e}_4^{2} &\vec{e}_4^{4} } \label{elem_mat}
	\end{align}
	\item The 4 point ${ DFT diagonal matrix}$ is defined as 
	\begin{align}
		\vec{D}_4 = diag\myvec{W_{8}^{0} & W_{8}^{1} & W_{8}^{2} & W_{8}^{3}}
	\end{align}
	\item Show that 
	\begin{equation}
		W_{N}^{2}=W_{N/2} \label{fft-3}
	\end{equation}
	%    \item Find $\vec{P}_6$.
	%    \item Find $\vec{D}_3$.
	\solution 
	From $\eqref{eq:twiddle}$,
	\begin{align}
		W_{N} = e^{-j2\pi/N}
	\end{align}
	Consider,
	\begin{align}
		W_{N}^{2} &= \brak{e^{-j2\pi/N}}^2 \\
		&= e^{-j2\pi/\brak{N/2}} \\
		&= W_{N/2}\label{result}
	\end{align}
	Hence proved.\\
	\item Show that 
	\begin{equation}
		\vec{F}_{4}=
		\begin{bmatrix}
			\vec{I}_{2} & \vec{D}_{2} \\
			\vec{I}_{2} & -\vec{D}_{2}
		\end{bmatrix}
		\begin{bmatrix}
			\vec{F}_{2} & 0 \\
			0 & \vec{F}_{2}
		\end{bmatrix}
		\vec{P}_{4}
	\end{equation}
	\solution From the eq $\eqref{elem_mat}$,
	\begin{align}
		\vec{P}_4 = \myvec{\vec{e}_4^{1} &\vec{e}_4^{3} &\vec{e}_4^{2} &\vec{e}_4^{4} }
	\end{align}
	Clearly $\vec{P}_4$ is an elementary matrix of $\vec{I}_{4}$, so on multiplication with a matrix it will interchange the rows/columns of matrix depending on positions of unit vectors.\\
	Generalising the condition ,
	\begin{align}
		\vec{P}_{N}^2 = \vec{I}_{N} \label{fft-4}
	\end{align}
	So it is similar to prove that, 
	\begin{equation} 
		\vec{F}_{4}\vec{P}_{4}=
		\begin{bmatrix}
			\vec{I}_{2} & \vec{D}_{2} \\
			\vec{I}_{2} & -\vec{D}_{2}
		\end{bmatrix}
		\begin{bmatrix}
			\vec{F}_{2} & 0 \\
			0 & \vec{F}_{2}   
		\end{bmatrix}                    
	\end{equation}
	Now from $\eqref{def:N-point_matrix}$,
	\begin{align}
		\vec{F}_{2} &= 
		\begin{bmatrix}
			W_2^{0.0} & W_2^{0.1} \\
			W_2^{1.0} & W_2^{1.1}
		\end{bmatrix} \\
		&= \begin{bmatrix}
			W_2^{0} & W_2^{0} \\
			W_2^{0} & W_2^{1}
		\end{bmatrix}
	\end{align}
	Using the result $\eqref{result}$, we can write
	\begin{align}
		\vec{F}_{2} &= 
		\begin{bmatrix}
			W_4^{0} & W_4^{0} \\
			W_4^{0} & W_4^{2}
		\end{bmatrix} 
	\end{align}
	And $\vec{D}_{2}$ is a diagonal matrix,
	\begin{align}
		\vec{D}_{2} &= diag\brak{W_4^0,W_4^1} \\
		&= diag\brak{1,W_4}
	\end{align}
	Then,
	\begin{align}
		\vec{D}_2\vec{F}_2 &=\begin{bmatrix}
			1 & 0 \\
			0 & W_4^{1}
		\end{bmatrix}  
		\begin{bmatrix}
			W_4^{0} & W_4^{0} \\
			W_4^{0} & W_4^{2}
		\end{bmatrix} \\
		&= \begin{bmatrix}
			W_4^{0} & W_4^{0} \\
			W_4^{1} & W_4^{3}
		\end{bmatrix}
	\end{align}
	And for $k \in \mathcal{N}$ and $N$ be a even integer we know that,
	\begin{align}
		W_{N}^{Nk} &= 1 \label{fft-1}\\
		W_{N}^{Nk + N/2} &= -1 \label{fft-2}
	\end{align}
	Using that we can write,
	\begin{align}
		-\vec{D}_2\vec{F}_2 &= \begin{bmatrix}
			W_4^{2} & W_4^{6} \\
			W_4^{3} & W_4^{9}
		\end{bmatrix}
	\end{align}
	And from $\eqref{def:N-point_matrix}$,
	\begin{align}
		\vec{F}_{4} &= \begin{bmatrix}
			W_4^0 &  W_4^0& W_4^0 & W_4^0  \\
			W_4^0 & W_4^1 & W_4^2 & W_4^3  \\
			W_4^0 & W_4^2 & W_4^4 &  W_4^6 \\
			W_4^0 & W_4^3 & W_4^6 & W_4^9   
		\end{bmatrix}
	\end{align}
	And 
	\begin{align}
		\vec{F}_{4}\vec{P}_{4} &= \begin{bmatrix}
			W_4^0 &  W_4^0& W_4^0 & W_4^0 \\
			W_4^0 & W_4^2 & W_4^1 & W_4^3  \\
			W_4^0 & W_4^4 & W_4^2 &  W_4^6 \\
			W_4^0 & W_4^6 & W_4^3 & W_4^9       
		\end{bmatrix}
	\end{align} 
	This is same as,
	\begin{align}
		\begin{bmatrix}
			\vec{F}_{2} & \vec{D}_{2}\vec{F}_{2} \\
			\vec{F}_{2} & -\vec{D}_{2}\vec{F}_{2}
		\end{bmatrix}\\
		\implies \begin{bmatrix}
			\vec{I}_{2} & \vec{D}_{2} \\
			\vec{I}_{2} & -\vec{D}_{2}
		\end{bmatrix}
		\begin{bmatrix}
			\vec{F}_{2} & 0 \\
			0 & \vec{F}_{2}   
		\end{bmatrix}   
	\end{align} 
	Hence proved.
	
	\item Show that 
	\begin{equation}
		\vec{F}_{N}=
		\begin{bmatrix}
			\vec{I}_{N/2} & \vec{D}_{N/2} \\
			\vec{I}_{N/2} & -\vec{D}_{N/2}
		\end{bmatrix}
		\begin{bmatrix}
			\vec{F}_{N/2} & 0 \\
			0 & \vec{F}_{N/2}
		\end{bmatrix}
		\vec{P}_{N}
	\end{equation}
\solution
For N even ;

We already know ;

\begin{align}
	\vec{F}_{N} = \sbrak{W_{N}^{mn}}, \quad 0 \le m,n \le N-1  	\\
	\vec{D}_{N}\vec{F}_{N} = \sbrak{W_{N}^{m.(2k+1)}}, \quad 0 \le m,k \le \frac{N}{2}-1  	
\end{align}

\begin{align}
	\vec{F}_{N}\vec{P}_{N}&=\begin{bmatrix}
		{W_{N}^{2mk}}&{W_{N}^{m.(2k+1)}}\\ {W_{N}^{2mk+Nk}}&{W_{N}^{m.(2k+1)+\frac{N}{2}.(2k+1)}}
	\end{bmatrix}  \nonumber \\
   &\quad \quad \quad \quad \quad 0 \le m,k \le \frac{N}{2}-1  \nonumber 	
\end{align}

From \eqref{fft-1} and \eqref{fft-2} ;

\begin{align}
	\vec{F}_{N}\vec{P}_{N}&=\begin{bmatrix}
		{W_{N}^{2mk}}&{W_{N}^{m.(2k+1)}}\\ {W_{N}^{2mk}}&-{W_{N}^{m.(2k+1)}}
	\end{bmatrix}  	
\end{align}
 
from \eqref{fft-3} ;

\begin{align}
	\vec{F}_{N}\vec{P}_{N}&=\begin{bmatrix}
		{W_{N/2}^{mk}}&{W_{N/2}^{m.(k+1/2)}}\\ {W_{N/2}^{mk}}&-{W_{N/2}^{m.(k+1/2)}}
	\end{bmatrix} 	\\
	\vec{F}_{N}\vec{P}_{N}&=\begin{bmatrix}
		\vec{F}_{N/2}&\vec{D}_{N/2}\vec{F}_{N/2}\\ \vec{F}_{N/2}&-\vec{D}_{N/2}\vec{F}_{N/2}
	\end{bmatrix}    	
\end{align}

Following \eqref{fft-4} ;

\begin{align}
	\vec{F}_{N}&=\begin{bmatrix}
		\vec{F}_{N/2}&\vec{D}_{N/2}\vec{F}_{N/2}\\ \vec{F}_{N/2}&-\vec{D}_{N/2}\vec{F}_{N/2}
	\end{bmatrix} \vec{P}_{N}   	
\end{align}

From above it follows ;

\begin{equation}
	\vec{F}_{N}=
	\begin{bmatrix}
		\vec{I}_{N/2} & \vec{D}_{N/2} \\
		\vec{I}_{N/2} & -\vec{D}_{N/2}
	\end{bmatrix}
	\begin{bmatrix}
		\vec{F}_{N/2} & 0 \\
		0 & \vec{F}_{N/2}
	\end{bmatrix}
	\vec{P}_{N}
\end{equation}

	\item Find 
	\begin{align}
		\vec{P}_4 \vec{x}
	\end{align}
\solution
   From \eqref{elem_mat},
   \begin{align}
   	\vec{P}_4&=\begin{bmatrix}
   		1&0&0&0\\0&0&1&0\\0&1&0&0\\0&0&0&1
   	\end{bmatrix}\\
   	\vec{x}&=\myvec{1\\2\\3\\4\\2\\1}
   \end{align}
   After proper zero padding of $\vec{P}_4$,
   \begin{align}
   	\vec{P}_4&=\begin{bmatrix}
   		1&0&0&0&0&0\\0&0&1&0&0&0\\0&1&0&0&0&0\\0&0&0&1&0&0\\0&0&0&0&0&0\\0&0&0&0&0&0
   	\end{bmatrix}\\
   	\vec{P}_4 \vec{x}&=\begin{bmatrix}
   		1&0&0&0&0&0\\0&0&1&0&0&0\\0&1&0&0&0&0\\0&0&0&1&0&0\\0&0&0&0&0&0\\0&0&0&0&0&0
   	\end{bmatrix}\myvec{1\\2\\3\\4\\2\\1}\\
   	&=\myvec{1\\3\\2\\4\\0\\0}
   \end{align}

	\item Show that 
	\begin{align}
		\label{eq:dft-mat-def}
		\vec{X} = \vec{F}_N \vec{x}
	\end{align}
	where $\vec{x}, \vec{X}$ are the vector representations of $x(n), X(k)$ respectively.\\
	\solution Given $\vec{x}, \vec{X}$ are the vector representations of $x(n), X(k)$ respectively.
	\begin{align}
		\vec{x}&=\begin{bmatrix}
			x(0)\\x(1)\\\vdots\\ x(N-1)
		\end{bmatrix}\\
		\vec{X}&=\begin{bmatrix}
			X(0)\\X(1)\\ \vdots\\ X(N-1)
		\end{bmatrix}\\
		\vec{F}_N &=\begin{bmatrix}
			1&1&!&\cdots&1\\1&W_N&W^2_N&\cdots&W_N^{(N-1)}\\1&W_N^2&W_N^4&\cdots&W^{2(N-1)}_N\\\vdots&\vdots&\vdots&\ddots&\vdots\\1&W_N^{N-1}&W_N^{2(N-1)}&\cdots&W_N^{(N-1)(N-1)}
		\end{bmatrix}
	\end{align}
	As \begin{align} 
		X(k)&=\sum_{n=0}^{N-1} x(n) e^{-j 2 \pi k n /N}
	\end{align}
	Upon linear transformation over k,
	\begin{align}
		\begin{bmatrix}
			X(0)\\X(1)\\ \vdots\\ X(N-1)
		\end{bmatrix}
		&=\begin{bmatrix}
			1&1&\cdots&1\\1&W_N&\cdots&W_N^{(N-1)}\\1&W_N^2&\cdots&W^{2(N-1)}_N\\\vdots&\vdots&\vdots&\vdots\\1&W_N^{N-1}&\cdots&W_N^{(N-1)(N-1)}
		\end{bmatrix}\begin{bmatrix}
			x(0)\\x(1)\\\vdots\\ x(N-1)
		\end{bmatrix}
	\end{align}
$\therefore$ $\vec{X} = \vec{F}_N \vec{x}$

	\item Derive the following Step-by-step visualisation  of
	8-point FFTs into 4-point FFTs and so on
	\begin{equation}
		\begin{bmatrix}
			X(0) \\ 
			X(1) \\ 
			X(2) \\ 
			X(3)
		\end{bmatrix}
		=
		\begin{bmatrix}
			X_{1}(0) \\ 
			X_{1}(1)\\ 
			X_{1}(2)\\
			X_{1}(3)\\
		\end{bmatrix}
		+
		\begin{bmatrix}
			W^{0}_{8} & 0 & 0 & 0\\
			0 & W^{1}_{8} & 0 & 0\\
			0 & 0 & W^{2}_{8} & 0\\
			0 & 0 & 0 & W^{3}_{8}
		\end{bmatrix}
		\begin{bmatrix}
			X_{2}(0) \\ 
			X_{2}(1) \\ 
			X_{2}(2) \\
			X_{2}(3)
		\end{bmatrix}
	\end{equation}
	\begin{equation}
		\begin{bmatrix}
			X(4) \\ 
			X(5) \\ 
			X(6) \\ 
			X(7)
		\end{bmatrix}
		=
		\begin{bmatrix}
			X_{1}(0) \\ 
			X_{1}(1)\\ 
			X_{1}(2)\\
			X_{1}(3)\\
		\end{bmatrix}
		-
		\begin{bmatrix}
			W^{0}_{8} & 0 & 0 & 0\\
			0 & W^{1}_{8} & 0 & 0\\
			0 & 0 & W^{2}_{8} & 0\\
			0 & 0 & 0 & W^{3}_{8}
		\end{bmatrix}
		\begin{bmatrix}
			X_{2}(0) \\ 
			X_{2}(1) \\ 
			X_{2}(2) \\
			X_{2}(3)
		\end{bmatrix}
	\end{equation}
	4-point FFTs into 2-point FFTs
	\begin{equation}
		\begin{bmatrix}
			X_{1}(0) \\ 
			X_{1}(1)\\ 
		\end{bmatrix}
		=
		\begin{bmatrix}
			X_{3}(0) \\ 
			X_{3}(1)\\ 
		\end{bmatrix}
		+
		\begin{bmatrix}
			W^{0}_{4} & 0\\
			0 & W^{1}_{4}
		\end{bmatrix}
		\begin{bmatrix}
			X_{4}(0) \\ 
			X_{4}(1) \\ 
		\end{bmatrix}
	\end{equation}
	\begin{equation}
		\begin{bmatrix}
			X_{1}(2) \\ 
			X_{1}(3)\\ 
		\end{bmatrix}
		=
		\begin{bmatrix}
			X_{3}(0) \\ 
			X_{3}(1)\\ 
		\end{bmatrix}
		-
		\begin{bmatrix}
			W^{0}_{4} & 0\\
			0 & W^{1}_{4}
		\end{bmatrix}
		\begin{bmatrix}
			X_{4}(0) \\ 
			X_{4}(1) \\ 
		\end{bmatrix}
	\end{equation}
	\begin{equation}
		\begin{bmatrix}
			X_{2}(0) \\ 
			X_{2}(1)\\ 
		\end{bmatrix}
		=
		\begin{bmatrix}
			X_{5}(0) \\ 
			X_{5}(1)\\ 
		\end{bmatrix}
		+
		\begin{bmatrix}
			W^{0}_{4} & 0\\
			0 & W^{1}_{4}
		\end{bmatrix}
		\begin{bmatrix}
			X_{6}(0) \\ 
			X_{6}(1) \\ 
		\end{bmatrix}
	\end{equation}
	\begin{equation}
		\begin{bmatrix}
			X_{2}(2) \\ 
			X_{2}(3)\\ 
		\end{bmatrix}
		=
		\begin{bmatrix}
			X_{5}(0) \\ 
			X_{5}(1)\\ 
		\end{bmatrix}
		-
		\begin{bmatrix}
			W^{0}_{4} & 0\\
			0 & W^{1}_{4}
		\end{bmatrix}
		\begin{bmatrix}
			X_{6}(0) \\ 
			X_{6}(1) \\ 
		\end{bmatrix}
	\end{equation}
	\begin{equation}
		P_{8}
		\begin{bmatrix}
			x(0) \\ 
			x(1) \\ 
			x(2) \\ 
			x(3) \\ 
			x(4) \\ 
			x(5) \\
			x(6) \\
			x(7)
		\end{bmatrix}
		= 
		\begin{bmatrix}
			x(0) \\ 
			x(2) \\ 
			x(4) \\ 
			x(6) \\
			x(1) \\ 
			x(3) \\ 
			x(5) \\
			x(7)
		\end{bmatrix}
	\end{equation}
	\begin{equation}
		P_{4}
		\begin{bmatrix}
			x(0) \\ 
			x(2) \\ 
			x(4) \\ 
			x(6) \\
		\end{bmatrix}
		= 
		\begin{bmatrix}
			x(0) \\ 
			x(4) \\ 
			x(2) \\
			x(6)
		\end{bmatrix}
	\end{equation}
	\begin{equation}
		P_{4}
		\begin{bmatrix}
			x(1) \\ 
			x(3) \\ 
			x(5) \\
			x(7)
		\end{bmatrix}
		= 
		\begin{bmatrix}
			x(1) \\ 
			x(5) \\ 
			x(3) \\ 
			x(7) \\
		\end{bmatrix}
	\end{equation}
	Therefore,
	\begin{equation}
		\begin{bmatrix}
			X_{3}(0) \\ 
			X_{3}(1)\\ 
		\end{bmatrix}
		= F_{2}
		\begin{bmatrix}
			x(0) \\ 
			x(4) \\ 
		\end{bmatrix}
	\end{equation}
	\begin{equation}
		\begin{bmatrix}
			X_{4}(0) \\ 
			X_{4}(1)\\ 
		\end{bmatrix}
		= F_{2}
		\begin{bmatrix}
			x(2) \\ 
			x(6) \\ 
		\end{bmatrix}
	\end{equation}
	\begin{equation}
		\begin{bmatrix}
			X_{5}(0) \\ 
			X_{5}(1)\\ 
		\end{bmatrix}
		= F_{2}
		\begin{bmatrix}
			x(1) \\ 
			x(5) \\ 
		\end{bmatrix}
	\end{equation}
	\begin{equation}
		\begin{bmatrix}
			X_{6}(0) \\ 
			X_{6}(1)\\ 
		\end{bmatrix}
		= F_{2}
		\begin{bmatrix}
			x(3) \\ 
			x(7) \\ 
		\end{bmatrix}
	\end{equation}
	\item For 
	\begin{align}
		\vec{x} = \myvec{1\\2\\3\\4\\2\\1}
		\label{eq:equation1}
	\end{align}
	compte the DFT  
	using 
	\eqref{eq:dft-mat-def}
	\solution \begin{align}
		\vec{F}_6&=\begin{bmatrix}
			1&1&1&1&1&1\\1&e^{-j \pi/3 }&e^{-j 2 \pi/3 }&e^{-j \pi }&e^{-j 4 \pi/3 }&e^{-j 5 \pi/3 }\\1&e^{-j 2 \pi/3 }&e^{-j 4 \pi/3 }&e^{-j 2 \pi }&e^{-j 8\pi/3 }&e^{-j 10 \pi/3 }\\1&e^{-j \pi }&e^{-j 2 \pi }&e^{-j 3 \pi }&e^{-j 4 \pi }&e^{-j 5 \pi }\\1&e^{-j 4 \pi/3 }&e^{-j 8 \pi/3 }&e^{-j 4 \pi }&e^{-j 16 \pi/3 }&e^{-j 20 \pi/3 }\\1&e^{-j 5 \pi/3 }&e^{-j 10 \pi/3 }&e^{-j 5 \pi }&e^{-j 20 \pi/3 }&e^{-j 25 \pi/3 }
		\end{bmatrix}
	\end{align}
	Using \eqref{eq:equation1},
	\begin{align}
		\vec{X}&=\vec{F}_6\vec{x}
	\end{align}
	\begin{align}
		\vec{X}=\begin{bmatrix}
			1&1&1&1&1&1\\1&e^{-j \pi/3 }&e^{-j 2 \pi/3 }&e^{-j \pi }&e^{-j 4 \pi/3 }&e^{-j 5 \pi/3 }\\1&e^{-j 2 \pi/3 }&e^{-j 4 \pi/3 }&e^{-j 2 \pi }&e^{-j 8\pi/3 }&e^{-j 10 \pi/3 }\\1&e^{-j \pi }&e^{-j 2 \pi }&e^{-j 3 \pi }&e^{-j 4 \pi }&e^{-j 5 \pi }\\1&e^{-j 4 \pi/3 }&e^{-j 8 \pi/3 }&e^{-j 4 \pi }&e^{-j 16 \pi/3 }&e^{-j 20 \pi/3 }\\1&e^{-j 5 \pi/3 }&e^{-j 10 \pi/3 }&e^{-j 5 \pi }&e^{-j 20 \pi/3 }&e^{-j 25 \pi/3 }
		\end{bmatrix} \myvec{1\\2\\3\\4\\2\\1}
	\end{align}
	\begin{align}
		=\begin{bmatrix}
			13\\-3.12-6.53j\\1j\\1.12-0.53j\\-1j\\1.12+0.53j
		\end{bmatrix}
	\end{align}
	\item Repeat the above exercise using the FFT
	after zero padding $\vec{x}$.\\
	\solution $\vec{x}$ after padding is 
	\begin{align}
		\myvec{1\\2\\3\\4\\2\\1\\0\\0}
	\end{align}
	Using 8-point fft ,
	\begin{align}
		\vec{F}_{8}=
		\begin{bmatrix}
			\vec{I}_{4} & \vec{D}_{4} \\
			\vec{I}_{4} & -\vec{D}_{4}
		\end{bmatrix}
		\begin{bmatrix}
			\vec{F}_{4} & 0 \\
			0 & \vec{F}_{4}
		\end{bmatrix}
		\vec{P}_{8}\\
		\vec{F}_{4}=
		\begin{bmatrix}
			\vec{I}_{2} & \vec{D}_{2} \\
			\vec{I}_{2} & -\vec{D}_{2}
		\end{bmatrix}
		\begin{bmatrix}
			\vec{F}_{2} & 0 \\
			0 & \vec{F}_{2}
		\end{bmatrix}
		\vec{P}_{4}\\
		\vec{F}_{2}=
		\begin{bmatrix}
			\vec{I}_{1} & \vec{D}_{1} \\
			\vec{I}_{1} & -\vec{D}_{1}
		\end{bmatrix}
		\begin{bmatrix}
			\vec{F}_{1} & 0 \\
			0 & \vec{F}_{1}
		\end{bmatrix}
		\vec{P}_{2}\\
		\vec{F_1}=\sbrak{1}
	\end{align}
	Calculating $\vec{F_2}$,
	\begin{align}
		\vec{F_2}&=\begin{bmatrix}
			\vec{F}_{1} & \vec{D_1F_1} \\
			\vec{F}_{1} & -\vec{D_1F_1}
		\end{bmatrix}\vec{P_2}\\
		&=\begin{bmatrix}1&1\\1&-1\end{bmatrix}
	\end{align}
	Calculating $\vec{F_4}$,
	\begin{align}
		\vec{D}_{2}=diag(1,W_4)
		=\begin{bmatrix}
			1&0\\0&-j
		\end{bmatrix}\\
		\vec{D_2F_2}=\begin{bmatrix}
			1&0\\0&-j
		\end{bmatrix}\begin{bmatrix}
			1&1\\1&-1
		\end{bmatrix}\\
		=\begin{bmatrix}
			1&1\\-j&j
		\end{bmatrix}\\
		\vec{F_4}=\begin{bmatrix}
			\vec{F}_{2} & \vec{D_2F_2} \\
			\vec{F}_{2} & -\vec{D_2F_2}
		\end{bmatrix}\vec{P_4}\\
		\vec{F_4}=\begin{bmatrix}
			1&0&1&1\\0&1&-j&j\\1&0&-1&-1\\0&1&j&-j
		\end{bmatrix}\begin{bmatrix}
			1&0&0&0\\0&0&1&0\\0&1&0&0\\0&0&0&1
		\end{bmatrix}\\
		=\begin{bmatrix}
			1&1&0&1\\0&-j&1&j\\1&-1&0&j\\0&j&1&-j
		\end{bmatrix}
	\end{align}
	Calculating $\vec{F_8}$,
	\begin{align}
		\vec{D_4}=diag\brak{1,W_8,W_8^2,W_8^3}\\
		=\begin{bmatrix}
			1&0&0&0\\0&\frac{1-j}{\sqrt{2}}&0&0\\0&0&-1&0\\0&0&0&\frac{-1-j}{\sqrt{2}}
		\end{bmatrix}\\
		\vec{D_4F_4}=\begin{bmatrix}
			1&0&0&0\\0&\frac{1-j}{\sqrt{2}}&0&0\\0&0&-1&0\\0&0&0&\frac{-1-j}{\sqrt{2}}
		\end{bmatrix}\begin{bmatrix}
			1&1&0&1\\0&-j&1&j\\1&-1&0&j\\0&j&1&-j
		\end{bmatrix}\\
		=\begin{bmatrix}
			1&1&0&1\\0&\frac{-1-j}{\sqrt{2}}&\frac{1-j}{\sqrt{2}}&\frac{1+j}{\sqrt{2}}\\-1&1&0&-j\\0&\frac{1-j}{\sqrt{2}}&\frac{-1-j}{\sqrt{2}}&\frac{-1+j}{\sqrt{2}}
		\end{bmatrix}
	\end{align}
	$F_8=\vec{ABP_8}$ where
	\begin{align}
		\vec{A}=\begin{bmatrix}
			1&0&0&0&1&0&0&0\\0&1&0&0&0&\frac{-1-j}{\sqrt{2}}&\frac{1-j}{\sqrt{2}}&\frac{1+j}{\sqrt{2}}\\0&0&1&0&-1&1&0&-j\\0&0&0&1&0&\frac{1-j}{\sqrt{2}}&\frac{-1-j}{\sqrt{2}}&\frac{-1+j}{\sqrt{2}}\\1&0&0&0&-1&0&0&0\\0&1&0&0&0&\frac{1+j}{\sqrt{2}}&\frac{-1+j}{\sqrt{2}}&\frac{-1-j}{\sqrt{2}}\\0&0&1&0&1&-1&0&j\\0&0&0&1&0&\frac{-1+j}{\sqrt{2}}&\frac{1+j}{\sqrt{2}}&\frac{1-j}{\sqrt{2}}
		\end{bmatrix}\\ \vec{B}=\begin{bmatrix}
			1&1&0&1&0&0&0&0\\0&-j&1&j&0&0&0&0\\1&-1&0&j&0&0&0&0\\0&j&1&-j&0&0&0&0\\0&0&0&0&-1&-1&0&-1\\0&0&0&0&0&j&-1&j\\0&0&0&0&-1&1&0&-j\\0&0&0&0&0&-j&-1&j
		\end{bmatrix}		
	\end{align}
	And $\vec{P_8}$ is a permutation matrix.
	\begin{align}
		\vec{X}=\begin{bmatrix}
			13\\-3.12-6.53j\\1j\\1.12-0.53j\\-1\\1.12+0.53j\\-1j\\-3.12+6.53j
		\end{bmatrix}
	\end{align}
	\item Write a C program to compute the 8-point FFT.
	\solution \begin{lstlisting} 
		wget https://github.com/karthik6281/Signal-Processing/tree/main/Assignment1/codes/7_13.c
		gcc 7_13.c -lm
	\end{lstlisting}
\end{enumerate}
\section{Exercises}
Answer the following questions by looking at the python code in Problem \ref{prob:output}.
\begin{enumerate}[label=\thesection.\arabic*]
	\item
	The command
	\begin{lstlisting}
		output_signal = signal.lfilter(b, a, input_signal)
	\end{lstlisting}
	in Problem \ref{prob:output} is executed through the following difference equation
	\begin{equation}
		\label{eq:iir_filter_gen}
		\sum _{m=0}^{M}a\brak{m}y\brak{n-m}=\sum _{k=0}^{N}b\brak{k}x\brak{n-k}
	\end{equation}
	%
	where the input signal is $x(n)$ and the output signal is $y(n)$ with initial values all 0. Replace
	\textbf{signal.filtfilt} with your own routine and verify.
	%
	\item Repeat all the exercises in the previous sections for the above $a$ and $b$.
	\item What is the sampling frequency of the input signal?
	\\
	\solution
	Sampling frequency(fs)=44.1kHZ.
	\item
	What is type, order and  cutoff-frequency of the above butterworth filter
	\\
	\solution
	The given butterworth filter is low pass with order=2 and cutoff-frequency=4kHz.
	%
	\item
	Modifying the code with different input parameters and to get the best possible output.
	%
\end{enumerate}
\begin{figure}[!ht]
	\centering
	\includegraphics[width=\columnwidth]{./figs/timecomparision.png}
	\caption{timecomparision.png}
	\label{fig-7.13}	
\end{figure}
\end{document}
