\documentclass[journal,12pt,twocolumn]{IEEEtran}

\usepackage{enumitem}
\usepackage{amsmath}
\usepackage{amssymb}
\usepackage{gensymb}
\usepackage{graphicx}
\usepackage{txfonts}         
\usepackage{listings}
\usepackage{lstautogobble}
\usepackage{mathtools}
\usepackage{bm}
\usepackage{hyperref}
\usepackage{polynom}
\usepackage{capt-of}
\usepackage{circuitikz}
\newcommand{\solution}{\noindent \textbf{Solution: }}
\providecommand{\pr}[1]{\ensuremath{\Pr\left(#1\right)}}
\providecommand{\brak}[1]{\ensuremath{\left(#1\right)}}
\providecommand{\mean}[1]{E\left[ #1 \right]}
\providecommand{\var}[1]{\mathrm{Var}\left[ #1 \right]}
\providecommand{\der}[1]{\mathrm{d} #1}
\providecommand{\gauss}[2]{\mathcal{N}\ensuremath{\left(#1,#2\right)}}
\providecommand{\mbf}{\mathbf}
\providecommand{\abs}[1]{\left\vert#1\right\vert}
\providecommand{\norm}[1]{\left\lVert#1\right\rVert}
\providecommand{\z}[1]{{\mathcal{Z}}\{#1\}}
\providecommand{\ztrans}{\overset{\mathcal{Z}}{ \rightleftharpoons}}
\providecommand{\system}[1]{\overset{\mathcal{#1}}{ \longleftrightarrow}}
\providecommand{\laplaceinv}[1]{{\mathcal{L}^{-1}\ensuremath{\left[#1\right]}}}
\providecommand{\parder}[2]{\frac{\partial}{\partial #2} \brak{#1}}

\let\StandardTheFigure\thefigure
\let\vec\mathbf

\numberwithin{equation}{section}
\renewcommand{\thefigure}{\theenumi}
\renewcommand\thesection{\arabic{section}}

\newcommand{\myvec}[1]{\ensuremath{\begin{pmatrix}#1\end{pmatrix}}}
\newcommand{\mydet}[1]{\ensuremath{\begin{vmatrix}#1\end{vmatrix}}}
\newcommand{\define}{\stackrel{\triangle}{=}}

\DeclareMathOperator*{\argmin}{arg\,min}
\DeclareMathOperator*{\argmax}{arg\,max}


\lstset {
	frame=single, 
	breaklines=true,
	columns=fullflexible,
	autogobble=true
}             



\begin{document}
	
	\title{ Digital Signal Processing  \\ 
		\Large EE3900
		%: Linear Systems and Signal Processing \\ \large Indian Institute of Technology Hyderabad \\ \vspace*{12pt} \textbf{Circuits and Transforms}
	}
	\author{Ravula Karthik \\ \normalsize AI21BTECH11024 \\ \vspace*{20pt} \normalsize \today  }
	\maketitle 
	\tableofcontents
	\section{Definitions}
	\begin{enumerate}[label=\arabic*.,ref=\thesection.\theenumi]
		\numberwithin{equation}{section}
		\numberwithin{figure}{section}
		\item The unit step function is 
		\begin{align}
			\label{eq:unit-step}
			u(t) =
			\begin{cases}
				1 & t > 0
				\\
				\frac{1}{2} & t = 0
				\\
				0 & t < 0
			\end{cases}
		\end{align}
		\item The Laplace transform of $g(t)$ is defined as 
		\begin{align}
			G(s) = \int_{-\infty}^{\infty} g(t) e^{-st}\, dt
		\end{align}
	\end{enumerate}
	
	\section{Laplace Transform}
	\begin{enumerate}[label=\arabic*.,ref=\thesection.\theenumi]
		\numberwithin{equation}{section}
		\item In the circuit, the switch S is connected to position P for a long time so that the charge on the capacitor
		becomes $q_1 \, \mu C$. Then S is switched to position Q.  After a long time, the charge on the capacitor is
		$q_2 \, \mu C$.
		\begin{figure}[!ht]
			\centering
			\includegraphics[width=\columnwidth]{figs/ckt.jpg}
			\caption{}
			\label{fig:ckt}
		\end{figure}
		\item Draw the circuit using latex-tikz.
		
		\solution The following code yields Fig.\ref{fig:qn}
		\begin{lstlisting}
			wget https://github.com/karthik6281/Signal-Processing/tree/master/CIRCUITS/Tikz%20Circuits/2_2.tex
		\end{lstlisting}
		\begin{figure}[!ht]
			\centering
			\input{Tikz Circuits/2_2.tex}
			\caption{Given Circuit}
			\label{fig:qn}
		\end{figure}
		
		\item Find $q_1$.
		
		\solution
		
		At steady state, which achieved when switch S is at P for long time capacoitor behaves as an open switch, hence current through capacitor is $0$,
		Let $i$ be the current flowing in the circuit at steady state. Applying KVL ,
		\begin{align}
			1-i-2i-2=0\\
			3i=-1 \Rightarrow i=\frac{-1}{3}A
		\end{align}
		Potential Difference across the capacitor at steady state is
		\begin{align}
			1-\brak{\frac{-1}{3}}=\frac{4}{3}V\\
			q_1=\frac{4}{3} \mu C
		\end{align}
		
		\begin{figure}
			\input{Tikz Circuits/2_3.tex}
			\caption{Before switching S to Q}
		\end{figure}
		
		\item Show that the Laplace transform of $u(t)$ is $\frac{1}{s}$ and find the ROC.
		\solution We know that Laplace Transform fo function $f(t)$ is given as $F(s)$,
		\begin{align}
			\label{eq:LaplaceTrans}
			F(s)&= \int_{0}^{\infty} f(t)e^{-st} \,dt \\
		\end{align}
		For $u(t)$, we have,
		\begin{align}
			F(s)&=\int_{0}^{\infty} u(t)e^{-st} \,dt
		\end{align}
		Using \eqref{eq:unit-step},
		\begin{align}
			F(s)&=\int_{0}^{\infty} u(t)e^{-st} \,dt\\
			&=\int_{0}^{\infty} e^{-st} \,dt\\
			&=-\brak{0-\frac{1}{s}}\\
			&=\frac{1}{s}
		\end{align}
		ROC is $ Re(s)>0$ since for $s>0$, $e^{-st}<\infty$ for $t \to \infty$
		\begin{figure}[!ht]
			\centering
			\includegraphics[width=\columnwidth]{figs/2_4.png}
			\caption{}
			\label{fig:roc1}
		\end{figure}
		\item Show that 
		\begin{align}
			e^{-at}u(t) \system{L} \frac{1}{s+a}, \quad a > 0
		\end{align}
		and find the ROC.\\
		\solution From \ref{eq:LaplaceTrans},
		\begin{align}
			F(s)&=\int_{0}^{\infty} u(t)e^{-at}e^{-st} \,dt\\
			&=\int_{0}^{\infty} u(t)e^{-\brak{s+a}t} \,dt\\
			&=\int_{0}^{\infty} e^{-\brak{s+a}t} \,dt\\
			&=-\brak{0-\frac{1}{s+a}}\\
			&=\frac{1}{s+a}
		\end{align}
		ROC is
		\begin{align}
			Re(s)+a>0 \Rightarrow  Re(s)>-a
		\end{align}
		\begin{figure}[!ht]
			\centering
			\includegraphics[width=\columnwidth]{figs/2_5.png}
			\caption{}
			\label{fig:roc2}
		\end{figure}
		\item Now consider the following resistive circuit transformed from 
		Fig. \ref{fig:ckt}
		\begin{figure}[!ht]
			\centering
			\includegraphics[width=\columnwidth]{figs/lap-ckt.jpg}
			\caption{}
			\label{fig:lap-ckt}
		\end{figure}
		where 
		\begin{align}
			u(t) \system{L} V_1(s)
			\\
			2u(t) \system{L} V_2(s)
		\end{align}
		Find the voltage across the capacitor $V_{C_0}(s)$.\\
		\solution
		\begin{align}
			R_{eff}=\frac{1}{1+\frac{1}{2}}
			=\frac{2}{3} \Omega\\
			V_{eff}=\frac{1}{1+\frac{1}{2}}
			=\frac{2}{3}V
		\end{align}
		%Effective Circuit in Laplacian Space is
		\begin{align}
			V_{C_0}(s)&=V_{S}(s)\frac{C_{0}}{C_{0}+R_{eff}}\\
			&=\brak{\frac{4}{3s}}\brak{\frac{\frac{1}{s}}{\frac{1}{s}+\frac{2}{3}}}\\
			\label{eq:laptr}
			&=\frac{3+4s}{3s\brak{s+\frac{3}{2}}}
		\end{align}
		\item Find $v_{C_0}(t)$.  Plot using python.\\
		\solution Running the following code gives the plot.
		\begin{lstlisting}
			wget https://github.com/karthik6281/Signal-Processing/tree/master/CIRCUITS/codes/2_7.py
		\end{lstlisting}
		
		Using \eqref{eq:laptr},
		\begin{align}
			\frac{3+4s}{3s\brak{s+\frac{3}{2}}}&=\frac{2}{3s}+\frac{2}{3(\frac{3}{2}+s)}
		\end{align}
		Apply inverse Laplacian Transform,
		\begin{align}
			V_{C_0}(s)\system{L^{-1}}V_{C_0}(t)\\
			\laplaceinv{V_{C_0}(s)}&=\laplaceinv{\frac{2}{3s}+\frac{2}{3(\frac{3}{2}+s)}}\\
			&=	\laplaceinv{\frac{2}{3s}}-\frac{2}{3}\laplaceinv{\frac{1}{\frac{3}{2}+s}}
		\end{align}
		Since,
		\begin{align}
			\laplaceinv{\frac1s}&=u(t)\\
			\laplaceinv{\frac{1}{s-a}}&=e^{at}u(t)
		\end{align}
		Using the above equations,
		\begin{align}
			V_{C_0}(t)=\frac{2}{3}\brak{ 1+e^{\frac{-3}{2} t}}u(t)
		\end{align}
		\begin{figure}[!ht]
			\centering
			\includegraphics[width=\columnwidth]{figs/2_7.png}
			\caption{Plot of $V_{C_0}(t)$}
			\label{fig:lap}
		\end{figure}
		\item Verify your result using ngspice.\\
		
		\solution Results obtained can be verified by running the following code.
		\begin{lstlisting}
			wget https://github.com/karthik6281/Signal-Processing/tree/master/CIRCUITS/codes/2_8.cir
		\end{lstlisting}
		And is plotted using the below code.
		\begin{lstlisting}
			wget https://github.com/karthik6281/Signal-Processing/tree/master/CIRCUITS/codes/2_8.py
		\end{lstlisting}
	
		\begin{figure}[!ht]
			\centering
			\includegraphics[width=\columnwidth]{figs/2_8.png}
			\caption{}
			\label{fig:ngspice}
		\end{figure}
		
		\item Obtain Fig. \ref{fig:lap} using the equivalent differential equation
		
		\solution 
		Using Kirchoff's junction law
		\begin{align}
			\frac{v_c(t) - v_1(t)}{R_1} + \frac{v_c(t) - v_2(t)}{R_2} + \frac{\der{q}}{\der{t}} = 0
		\end{align}
		where $q(t)$ is the charge on the capacitor
		
		On taking the Laplace transform on both sides of this equation
		\begin{align}
			\frac{V_c(s) - V_1(s)}{R_1} + \frac{V_c(s) - V_2(s)}{R_2} + \brak{sQ(s) - q(0^-)} = 0
		\end{align}
		
		But $q(0^-) = 0$ and 
		\begin{align}
			q(t) &= C_0v_c(t) \\
			\implies Q(s) &= C_0V_c(s)
		\end{align}
		
		Thus
		\begin{align}
			&\frac{V_c(s) - V_1(s)}{R_1} + \frac{V_c(s) - V_2(s)}{R_2} + sC_0V_c(s) = 0 \\
			\implies &\frac{V_c(s) - V_1(s)}{R_1} + 	\frac{V_c(s) - V_2(s)}{R_2} + \frac{V_c(s) - 0}{\frac{1}{sC_0}} = 0 
		\end{align}
		
		which is the same equation as the one we obtained from Fig. \ref{fig:lap}
		
	\end{enumerate}
	
	\section{Initial Conditions}
	\begin{enumerate}[label=\arabic*.,ref=\thesection.\theenumi]
		\numberwithin{equation}{section}
		\item Find $q_2$ in Fig. \ref{fig:ckt}.\\
		\solution At steady state capacitor behaves as an open switch. Hence $V_{C_0}=V_{1 \Omega}$.\\
		Let $i$ be the current in the circuit. Using KVL,
		\begin{align}
			2-2i-i=0 \implies i=\frac{2}{3}\\
			V_{1 \Omega}=i \times 1= \frac{2}{3} V\\
			V_{C_0}=\frac{q_2}{C_0}=V_{1 \Omega}=\frac{2}{3}\\
			\implies q_2=\frac{2}{3} \mu C
		\end{align}
		\item Draw the equivalent $s$-domain resistive circuit when S is switched to position Q.  Use variables $R_1, R_2, C_0$ for the passive elements.
		Use latex-tikz.
		\label{prob:init}
		\\\solution 
		\begin{figure}[!ht]
			\centering
			\input{Tikz Circuits/3_2.tex}
			\caption{After switching S to Q}
			\label{fig:sq}
		\end{figure}
		\item $V_{C_0}(s)$ = ? \\
		\solution Let voltage across capacitor be $V$. Using KCL at node in Fig. \ref{fig:sq}
		\begin{align}
			\frac{V - 0}{R_1} + \frac{V - \frac{2}{s}}{R_2} + sC_0\brak{V - \frac{4}{3s}} = 0 \\
			\implies V_{C_0}(s) = \frac{\frac{2}{sR_2} + \frac{4C_0}{3}}{\frac{1}{R_1} + \frac{2}{R_2} + sC_0}
			\label{eq:v2-s}
		\end{align} 
		\item $v_{C_0}(t)$ = ? Plot using python.\\
		\solution Running the following code gives the plot.
		\begin{lstlisting}
			wget https://github.com/karthik6281/Signal-Processing/tree/master/CIRCUITS/codes/3_4.py
		\end{lstlisting}
		From \eqref{eq:v2-s},
		\begin{align}
			&V_{C_0}(s) = \frac{4}{3}\brak{\frac{1}{\frac{1}{C_0}\brak{\frac{1}{R_1} + \frac{1}{R_2}}+s}} \nonumber \\
			&+ \frac{2}{R_2\brak{\frac{1}{R_1} +\frac{1}{R_2}}}\brak{\frac{1}{s} - \frac{1}{\frac{1}{C_0}\brak{\frac{1}{R_1} + \frac{1}{R_2}} + s}}
		\end{align}
		Taking an inverse Laplace Transform,
		\begin{align}
			&v_{C_0}(t) = \frac{4}{3}e^{-\brak{\frac{1}{R_1} + \frac{1}{R_2}}\frac{t}{C_0}}u(t) \nonumber \\ 
			&+ \frac{2}{R_2\brak{\frac{1}{R_1}+\frac{1}{R_2}}}\brak{1 - e^{-\brak{\frac{1}{R_1} + \frac{1}{R_2}}\frac{t}{C_0}}}u(t)
		\end{align}
		Substituting values gives
		\begin{align}
			v_{C_0}(t) = \frac{2}{3}\brak{1 +e^{-\brak{1.5 \times 10^6}t}}u(t)
			\label{eq:v2-t}
		\end{align}
		\begin{figure}[!ht]
			\centering
			\includegraphics[width=\columnwidth]{figs/3_4.png}
			\caption{Plot of $V_{C_0}(t)$}
			%\label{fig:lap-ckt}
		\end{figure}
		\begin{figure}[!ht]
			\centering
			\includegraphics[width=\columnwidth]{figs/3_5.png}
			\caption{ngspice plot of $V_{C_0}(t)$} 
			\label{fig:ngspice2}
		\end{figure}
		\item Verify your result using ngspice.\\
		\solution Results obtained can be verified by running the following code.
		\begin{lstlisting}
			wget https://github.com/karthik6281/Signal-Processing/tree/master/CIRCUITS/codes/3_5.cir
		\end{lstlisting}
		Runningn the below code plots the figure \ref{fig:ngspice2}, and verifies our result.
		\begin{lstlisting}
			wget https://github.com/karthik6281/Signal-Processing/tree/master/CIRCUITS/codes/3_5.py
		\end{lstlisting}
		
		
		
		\item Find $v_{C_0}(0-), v_{C_0}(0+)$ and  $v_{C_0}(\infty) $.\\
		\solution From the initial conditions,
		\begin{align}
			v_{C_0}(0-) = \frac{q_1}{C} = {\frac{4}{3}}{V}
		\end{align}
		Using \eqref{eq:v2-t},
		\begin{align}
			v_{C_0}(0+) &= \lim_{t \to 0+}v_{C_0}(t) = {\frac{4}{3}}{V} \\
			v_{C_0}(\infty) &= \lim_{t \to \infty}v_{C_0}(t) = {\frac{2}{3}}{V}
		\end{align}
		
		\item Obtain Fig. \ref{prob:init} using the equivalent differential equation
		
		\solution Using Kirchoff's junction law
		\begin{align}
			\frac{v_c(t) - 0}{R_1} + \frac{v_c(t) - v_2(t)}{R_2} + \frac{\der{q}}{\der{t}} = 0
		\end{align}
		where $q(t)$ is the charge on the capacitor
		
		On taking the Laplace transform on both sides of this equation
		\begin{align}
			\frac{V_c(s) - 0}{R_1} + \frac{V_c(s) - V_2(s)}{R_2} + \brak{sQ(s) - q(0^-)} = 0
		\end{align}
		
		But $q(0^-) = \frac43 C_0$ and 
		\begin{align}
			q(t) &= C_0v_c(t) \\
			\implies Q(s) &= C_0V_c(s)
		\end{align}
		
		Thus
		\begin{align}
			&\frac{V_c(s) - 0}{R_1} + \frac{V_c(s) - V_2(s)}{R_2} + \brak{sC_0V_c(s) - \frac43 C_0} = 0 \\
			\implies &\frac{V_c(s) - 0}{R_1} + 	\frac{V_c(s) - V_2(s)}{R_2} + \frac{V_c(s) - \frac{4}{3s}}{\frac{1}{sC_0}} = 0 
		\end{align}
		
		which is the same equation as the one we obtained from Fig. \ref{prob:init}
	\end{enumerate}
	
	\section{Bilinear Transform}
	\begin{enumerate}[label=\thesection.\arabic*.,ref=\thesection.\theenumi]
		\item In Fig. \ref{fig:ckt}, consider the case when $S$ is switched to $Q$ right in the beginning. Formulate the differential equation
		
		\solution 
		\begin{figure}[!ht]
			\centering
			\input{Tikz Circuits/4_1.tex}
			\caption{Switch S connected to Q initially}
			\label{fig:4.1}
		\end{figure}
		Considering KCL on the circuit \ref{fig:4.1}, we get the differential equuation as 
		\begin{align}
			\label{eq:diff.4.1}
			&\frac{v_c(t) - 0}{R_1} + \frac{v_c(t) - v_2(t)}{R_2} + \frac{\der{q}}{\der{t}} = 0 \\
			\implies &\frac{v_c(t)}{R_1} + \frac{v_c(t) - v_2(t)}{R_2} + C_0\frac{\der{v_c}}{\der{t}} = 0
		\end{align}
		Here we have $q(0)=0$, since initially the capacitor is uncharged.
		
		\item Find $H(s)$ considering the ouput voltage at the capacitor
		
		\solution Applying laplace transform to the equation \eqref{eq:diff.4.1}, we get
		\begin{align}
			&\frac{V_c(s)}{R_1} + \frac{V_c(s) - V_2(s)}{R_2} + \mathcal{L}(\frac{\der{q}}{\der{t}})= 0 \\
			&\frac{V_c(s)}{R_1} + \frac{V_c(s) - V_2(s)}{R_2} + sQ(s) - q(0) = 0 \\
			&\frac{V_c(s)}{R_1} + \frac{V_c(s) - V_2(s)}{R_2} + sQ(s) - 0 = 0 \\
			\implies &V_c(s) \brak{\frac{1}{R_1} + \frac{1}{R_2}} + sC_0V_c(s) = \frac{V_2(s)}{R_2} \\
			\implies &\frac{V_c(s)}{V_2(s)} = \frac{\frac{1}{R_2}}{\frac{1}{R_1} + \frac{1}{R_2} + sC_0}
		\end{align}
		Here, $Q(s)$ is the laplace transform of q, $V_c(s)$ is laplace transform of $v_c(t)$.
		Hence, the transform function($H(s)$) is 
		\begin{align}
			H(s) &= \frac{V_c(s)}{V_2(s)} \\
			&=\frac{\frac{1}{R_2C_0}}{s + \frac{1}{R_1C_0} + \frac{1}{R_2C_0}}
		\end{align}
		Substituting values of $R_1=1\Omega$,$R_2=2\Omega$ and $C_0=1\mu F$, we get,
		\begin{align}
			H(s) &= \frac{0.5}{s 10^{-6} + 1.5}\\
			\label{eq:Hs}
			\implies H(s)&=\frac{5 \times 10^5}{s + 1.5 \times 10^6}
		\end{align}
		
		\begin{figure}[!ht]
			\centering
			\includegraphics[width=\columnwidth]{./figs/4_2.png}
			\caption{ngspice plot of $H(t)$}
			\label{fig-4.2}
		\end{figure}
		The following python code plots the figure \ref{fig-4.2}
		\begin{lstlisting}
			wget https://github.com/karthik6281/Signal-Processing/tree/master/CIRCUITS/codes/4_2.py
		\end{lstlisting}
		%Applying inverse Laplace transform to equation \eqref{eq:Hs}, we get,
		%\begin{align}
		%H(t)=5\times 10^{5} e^{-1.5\times 10^{6}}
		%H(t)=\frac{1}{3} \frac{(1 - (1 - 7.5 10^5 t )}{(1 + 7.5 10^{5} t)}^{10^{7}n}	
		%\end{align}
		\item Plot $H(s)$.  What kind of filter is it?\\
		\solution THe below python code plots the Figure \ref{fig-4.3}
		\begin{lstlisting}
			wget https://github.com/karthik6281/Signal-Processing/tree/master/CIRCUITS/codes/4_3.py
		\end{lstlisting}
		\begin{figure}[!ht]
			\centering
			\includegraphics[width=\columnwidth]{./figs/4_3.png}
			\caption{Plot of $H(s)$}
			\label{fig-4.3}	
		\end{figure}
		Considering the frequency-domain transfer function ($H(s=e^{j\omega})$), from \eqref{eq:Hs},we get
		\begin{align}
			\label{eq:Hw}
			H(s=j\omega) &= \frac{5 \times 10^5}{j\omega + 1.5 \times 10^6} \\
			\implies \abs{H(s=j\omega)} &= \frac{5 \times 10^5}{\sqrt{\omega^2 + 2.25\times10^{12}}}
		\end{align}
		Clearly from \eqref{eq:Hw}, as $\omega$ increases, $H(s=j\omega)$ decreases(inverse proportionality). When high frequency signals( large values of $\omega$) pass through this transfer function ($H(s=j\omega$)), they become negligible, which results in removing high frequnecy signals and allowing only low frequency signal to pass. Hence, this is a low-pass filter.
		\item Using trapezoidal rule for integration, formulate the difference equation by considering 
		\begin{align}
			y(n) = y(t)\vert_{t=n}
		\end{align}
		\solution
		In the equation \eqref{eq:diff.4.1}, we have
		\begin{align}
			\frac{\der{q}}{\der{t}}&=C_0\frac{\der{v_c}}{\der{t}}\\
			v_2(t)&=2u(n)
		\end{align} 
		Hence,
		\begin{align}
			&\frac{v_c(t)}{R_1} + \frac{v_c(t) - v_2(t)}{R_2} + C_0\frac{\der{v_c}}{\der{t}} = 0 \\
			\implies &C_0\frac{\der{v_c}}{\der{t}} = \frac{2u(t)-v_c(t)}{R_2} - \frac{v_c(t)}{R_1} \\
			\implies &\frac{\der{v_c}}{\der{t}} = \frac{2u(t)-v_c(t)}{R_2 C_0} - \frac{v_c(t)}{R_1 C_0} \\
			\label{eq:trap}
			\implies &\left.v_c(t)\right|_{t=n}^{n+1} = \int_{n}^{n+1} \brak{\frac{2u(t)-v_c(t)}{R_2C_0} - \frac{v_c(t)}{R_1C_0}} \der{t}
		\end{align}
		
		From trapezoidal rule of integration
		\begin{align}
			\label{eq:traprule}
			\int_a^b f(t) \der{t} \approx \frac{b-a}{2} (f(a) + f(b))
		\end{align}
		Apply \eqref{eq:traprule}, to the RHS of the equation \eqref{eq:trap}, we get,
		\begin{align}
			\int_{n}^{n+1} \frac{2u(t)-v_c(t)}{R_2C_0} &- \frac{v_c(t)}{R_1C_0} \der{t}=\frac{1}{R_2C_0}\brak{u(n)+u(n+1)}\\&- \frac12(y(n+1) + y(n))\brak{\frac{1}{R_1C_0} + \frac{1}{R_2C_0}}
		\end{align}
		Considering $y(t) = v_c(t)$, from \eqref{eq:trap}, we get,
		\begin{multline}
			y(n+1) - y(n) = \frac{1}{R_2C_0}\brak{u(n)+u(n+1)} \\
			- \frac12(y(n+1) + y(n))\brak{\frac{1}{R_1C_0} + \frac{1}{R_2C_0}}
		\end{multline}
		Thus, the difference equation is
		\begin{multline}
			\implies y(n+1) \brak{1 + \frac{1}{2R_1C_0} + \frac{1}{2R_2C_0}} \\= y(n) \brak{1 - \frac{1}{2R_1C_0} - \frac{1}{2R_2C_0}} \\+ \frac{1}{R_2C_0}\brak{u(n)+u(n+1)}
		\end{multline}
		
		\item Find $H(z)$\\
		
		\solution Let $\z{y(n)} = Y(z)$
		
		On taking $\mathcal{Z}$-transform on both sides of the difference equation, we get,
		\begin{multline}
			zY(z)\brak{1 + \frac{1}{2R_1C_0} + \frac{1}{2R_2C_0}} \\= Y(z)\brak{1 - \frac{1}{2R_1C_0} - \frac{1}{2R_2C_0}} \\+ \frac{1}{R_2C_0} \brak{\frac{1}{1-z^{-1}} + \frac{z}{1-z^{-1}}}
		\end{multline}
		\begin{multline}
			Y(z)\brak{z + \frac{z}{2R_1C_0} + \frac{z}{2R_2C_0} - 1 + \frac{1}{2R_1C_0} + \frac{1}{2R_2C_0}} \\
			= \frac{1}{R_2C_0} \frac{1+z}{1-z^{-1}}
		\end{multline}
		Here, since $v_2(t) = 2 \forall t \ge 0\\$ \\
		Initial voltage is given as,
		\begin{align}
			\implies x(n) &= 2u(n) \\
			\implies X(z) &= \frac{2}{1-z^{-1}} &&\abs{z} > 1
		\end{align}
		Thus, the transfer function in $z$-domain is
		\begin{align}
			H(z) &= \frac{Y(z)}{X(z)} \\
			&= \frac{\frac{1+z}{2R_2C_0}}{z + \frac{z}{2R_1C_0} + \frac{z}{2R_2C_0} - 1 + \frac{1}{2R_1C_0} + \frac{1}{2R_2C_0}} \\
			&= \frac{\frac{1 + z^{-1}}{2R_2C_0}}{1 + \frac{1}{2R_1C_0} + \frac{1}{2R_2C_0} - z^{-1} + \frac{z^{-1}}{2R_1C_0} + \frac{z^{-1}}{2R_2C_0}}
		\end{align}
		Substituting the values of $R_1$,$R_2$ and $C_0$, we get,
		\begin{align}
			\label{eq:4.5}
			H(z) &= \frac{2.5\times10^5 (1+z^{-1})}{7.5\times10^5 + 1 + (7.5\times10^5 - 1)z^{-1}}
		\end{align}
		Where, ROC of $H(z)$ is,
		\begin{align}
			\abs{z} &> 1 \cap \abs{z}>\abs{\frac{7.5\times10^5 - 1}{7.5\times10^5 + 1}} \\
			\implies \abs{z} &> 1
		\end{align}
		
		\item How can you obtain $H(z)$ from $H(s)$?
		
		\solution The $Z$-transform can be obtained from the Laplace transform by the substitution
		\begin{align}
			s &= \frac{2}{T} \frac{1-z^{-1}}{1+z^{-1}}
		\end{align}
		where $T$ is the sampling time period, used in the trapezoidal rule. Here, its value is $1$. This known as the bilinear transform.\\
		From \eqref{eq:trap}, we have,
		\begin{align}
			H(z) &= \frac{\frac{1}{R_2C_0}}{2\frac{1-z^{-1}}{1+z^{-1}} + \frac{1}{R_1C_0} + \frac{1}{R_2C_0}} \\
			&= \frac{\frac{1 + z^{-1}}{2R_2C_0}}{1-z^{-1}	 + \brak{\frac{1}{2R_1C_0} + \frac{1}{2R_2C_0}}(1 + z^{-1})} \\
			&= \frac{\frac{1 + z^{-1}}{2R_2C_0}}{1 + \frac{1}{2R_1C_0} + \frac{1}{2R_2C_0} - z^{-1} + \frac{z^{-1}}{2R_1C_0} + \frac{z^{-1}}{2R_2C_0}} \\
			\label{eq:4.6}
			&= \frac{2.5\times10^5 (1+z^{-1})}{7.5\times10^5 + 1 + (7.5\times10^5 - 1)z^{-1}}
		\end{align}
		Here, this result obtained in the equation \eqref{eq:4.6} is same as the result we obtained in \eqref{eq:4.5}
		
		\item Find $y(n)$ from $H(z)$ and verify whether $y(n) = y(t)|_{t=n}$\\
		\solution 
		We know that, 
		\begin{align}
			Y(z) &= H(z)X(z) \\
			&= \brak{\frac{2.5\times10^5 (1+z^{-1})}{7.5\times10^5 + 1 + (7.5\times10^5 - 1)z^{-1}}} \frac{2}{1-z^{-1}} \\
			\label{eq:4.7}
			&= \frac{\frac{2}{3}}{1-z^{-1}} -\frac{\frac{2}{3}}{7.5\times10^5 + 1 + (7.5\times10^5 - 1)z^{-1}}
		\end{align}
		Let ROC be $\abs{z}>1$, we know that,
		\begin{align}
			\frac{1}{1-z^{-1}} &\system{Z} u(n)\\
			\frac{1}{1-a z^{-1}} &\system{Z} a^{n} u(n)
		\end{align}
		Applying inverse $\mathcal{Z}$ to the equation \eqref{eq:4.7}, we get 
		\begin{align}
			y(n) &= \frac{2}{3}u(n) - \frac{2}{3}\frac{1}{7.5\times10^5 + 1}\brak{-\frac{7.5\times10^5 - 1}{7.5\times10^5 + 1}}^nu(n) \\
			\label{eq:4.7.1}
			&= \frac{2}{3} \brak{1 - \frac{(1-7.5\times10^5)^n}{(1+7.5\times10^5)^{n+1}}}u(n)
		\end{align}
		%If we are sampling the signal at intervals of $T \ll 10^{-5}$, say $\SI[parse-numbers=false]{10^{-7}}{\second}$, i.e., $n = 10^{-7}, 2\times10^{-7},\ldots$
		Applying binomial theorem to the equation \eqref{eq:4.7.1}, ($(1+x)^{n} \approx 1+nx$ for $x \ll 1$), we get,
		\begin{align}
			\label{eq:4.7.3}
			y(n) &\approx \frac{2}{3} \brak{1 - \frac{1-7.5\times10^5n}{1+7.5\times10^5n}}u(n)
		\end{align}
		Now, consider $Y(s)$, 
		\begin{align}
			Y(s) &= H(s)X(s) \\
			&= \frac{5\times10^5}{s+1.5\times10^6} \frac{2}{s} \\
			\label{eq:4.7.2}
			&= \frac{10^6}{1.5\times10^6} \brak{\frac{1}{s} - \frac{1}{s+1.5\times10^6}}
		\end{align}
		We know that ,
		Let ROC be $\abs{z}>1$, we know that,
		\begin{align}
			\frac{1}{s} &\system{L} 1 \quad  \Re(s)>0\\
			\frac{1}{s+a} &\system{L} e^{-at} \quad \Re(s)>-a
		\end{align}
		Consider ROC as $\Re(s)>0$, applying inverse laplace transform to the equation \eqref{eq:4.7.2}, we get,  
		\begin{align}
			y(t) = \frac{2}{3}\brak{1 - e^{-1.5\times10^6t}}u(t)
		\end{align}
		But for $t \ll 10^-6$, we have
		\begin{align}
			e^{-1.5\times 10^6 t} &= \frac{e^{-0.75 \times 10^6 t}}{e^{0.75 \times 10^6 t}} \\
			&\approx \frac{1-7.5\times10^5t}{1+7.5\times10^5t}  
		\end{align}
		Therefore
		\begin{align}
			\label{eq:4.7.4}
			y(t) &\approx \frac{2}{3} \brak{1 - \frac{1-7.5\times10^5t}{1+7.5\times10^5t}} u(t) 
		\end{align}
		From equations \eqref{eq:4.7.3} and \eqref{eq:4.7.4}, we have,
		\begin{align}
			\therefore y(n) &= y(t)|_{t=n}
		\end{align}
		Hence verified.
		\begin{figure}[!ht]
			\centering
			\includegraphics[width=\columnwidth]{./figs/4_7.png}
			\caption{Plot of $y(t)$,$y(n)$ and ngspice}
			\label{fig:4.7}	
		\end{figure}
		The following python plots the graph \ref{fig:4.7} of $V_C(t)$.
		\begin{lstlisting}
			wget https://github.com/karthik6281/Signal-Processing/tree/master/CIRCUITS/codes/4_7.py
		\end{lstlisting}
		
	\end{enumerate}
	
\end{document}
